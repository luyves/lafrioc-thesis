\section{Conclusiones}\label{Conclusiones}

Al final de todo, se logra pues, publicar un artículo en una revista arbitrada como resultado del
trabajo detallado en este informe. El artículo pierde en un sentido y gana en otro con respecto a la
idea original. Durante el primer arbitraje y las respectivas correcciones se pierde la discusión
histórica y filosófica de las desigualdades de Bell y los estados entrelazados, además de algunos
ejemplos que se creía esclarecerían la idea tras las desigualdades de Bell. Sin embargo, al
deshacerse de esa parte, fue necesario ampliar el contenido acerca del experimento, se agrega una
parte explicando cómo puede obtenerse cualquiera de los 4 estados de Bell a partir de la disposición
original del experimento y el uso de placas retardadoras de fase.

El artículo en su estado final puede fácilmente cumplir con su labor de manual para llevar a cabo
el experimento. Además de citar fuentes útiles tanto para la discusión teórica de las desigualdades
como para su discusión filosófica. En el artículo se hace una clara distinción entre los resultados
que predice cualquier teoría de variables ocultas locales en este experimento y las predicciones de
la mecánica cuántica. Y por último se obtiene un parámetro de Bell que claramente viola la
desigualdad de CHSH, probando así experimentalmente una vez más el teorema de Bell.