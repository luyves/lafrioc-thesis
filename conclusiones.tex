\section{Conclusiones y perspectivas}\label{Conclusiones}

Este trabajo se enfocó en dejar listo un sistema de adquisición y procesamiento de datos para pares de fotones generados a partir de experimentos en átomos fríos de rubidio. Consistió en montar el sistema óptico de bombeo de luz y el sistema óptico de adquisición de captura, pensados para experimentos de mezclado de cuatro ondas.

Se escribió una paquetería completa en Python para procesar etiquetas temporales de eventos generados por fotodiodos de avalancha. Este programa se conecta con el módulo de cuentas \textit{id800}, coordina la comunicación del instrumento con la computadora y permite al usuario visualizar los datos generados en tiempo real, para después guardarlos en un servidor. El desarrollo de este software propio permitió quitarse de la limitante de tener que usar el programa proporcionado por el fabricante, logrando adaptar la funcionalidad del \textit{id800} a las necesidades del experimento y no al revés.

Además de esta paquetería para adquisición de datos, se desarrollaron dos sistemas útiles para el experimento de mezclado de cuatro ondas y para el Laboratorio en general. Estos sistemas fueron un perfilómetro para haces de luz Gaussianos y un circuito AND para controlar la señal de salida de los APDs.

Se realizó un experimento sencillo para verificar que el sistema de adquisición funcionara adecuadamente. Este experimento consistió en medir la función de correlación de segundo orden $\g{2}(\tau)$ para una fuente clásica de luz coherente. Se obtuvo un valor de $\g{2}(0) = 1.0006\pm0.0012$ y un promedio de $\overline{\g{2}} = 1.0007\pm0.0014$, que se encuentran en excelente concordancia con la teoría. Esto demostró el buen funcionamiento del programa de adquisición de datos, que actualmente ya se usa para medir correlaciones en fotones individuales generados por mezclado de cuatro ondas.

Finalmente, se escribió una guía para usar la paquetería de adquisición de datos y su interfaz gráfica. Esta guía se puede encontrar en el Apéndice \ref{apendice_TDC}, y la paquetería se encuentra en \cite{github}. El siguiente paso para la continuación de este trabajo sería integrar el programa de adquisición de datos al sistema de control que se encuentra en desarrollo para experimentos de mezclado de cuatro ondas.

Como objetivo secundario, se buscó una primer caracterización de la nube atómica dentro de la MOT en función de sus parámetros experimentales. Se realizaron una serie de mediciones que documentan las propiedades espectroscópicas de nuestra MOT con todos los láseres y campos magnéticos relevantes encendidos. Se midió la densidad óptica de la nube en función del gradiente del campo magnético de atrapamiento, la desintonía e intensidad de los haces de enfriamiento y la presión dentro de la cámara de vacío. Sin embargo, un análisis posterior encontró que estas medidas están sobreestimadas por un valor de $\sim$ 3 y no son los valores reales de la densidad óptica de la nube. A pesar de esto, estos resultados constituyen una primer referencia experimental para el control de la densidad óptica de la MOT del Laboratorio.

Además, ya se empezó a realizar la caracterización de la densidad óptica de la nube pulsando los haces de enfriamiento. Esto sirve con el propósito de obtener medidas de la densidad óptica sin el pico dispersivo observado en los espectros de absorción de la sección \ref{resultados_espectro}. Una explicación de los fenónemos detrás de estos picos no se incluye en este trabajo. Sin embargo, efectos de EIT en gases de rubidio han sido usados para anclar la frecuencia de láseres a algunas resonancias atómicas\cite{Becerra:09}, por lo que podría ser de interés estudiar más a detalle su estructura.