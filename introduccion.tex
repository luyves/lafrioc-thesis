\section*{Introducción}\label{introduccion}
\addcontentsline{toc}{section}{Introducción}
%La invención de técnicas de enfriamiento atómico en los 80s\cite{PhysRevLett.24.156,PhysRevLett.59.2631} permitió manipular y monitorear la evolución temporal de ensambles atómicos a ultra-bajas temperaturas. Casi todos los datos experimentales de estos átomos fríos se consiguen por medio de observaciones ópticas\cite{901989,1742-6596-80-1-012047}; la \textit{óptica cuántica} estudia la interacción entre luz y materia desde una perspectiva cuántica.
%
%Estos sistemas de átomos fríos confinados tienen un gran potencial como herramientas de estudio en correlaciones y entrelazamiento\cite{Lukens:15}, con mucho interés en implementarlos para procesamiento en información cuántica\cite{PhysRevLett.101.190501}. Para esto, es indispensable poder crear, manipular y medir los estados cuánticos de estos sistemas. Una transmisión eficiente de información entre dos sistemas atómicos requiere una interacción fuerte entre átomos y fotones.
%
%El Laboratorio de Átomos Fríos y Óptica Cuántica del Instituto de Física de la UNAM --- parte del Laboratorio Nacional de Materia Cuántica --- tiene como propósito hacer investigación en metrología e información cuántica, por medio del estudio de sistemas cuánticos ópticos y materiales.
%
%El experimento principal del Laboratorio es el de generación de pares de fotones (o \textit{bifotones}) por medio de un proceso no lineal llamado \textit{mezclado de cuatro ondas} --- o FWM por sus siglas en inglés. En el Laboratorio, este proceso de FWM es realizado actualmente en una muestra de átomos de rubidio calientes, y próximamente se realizará también en una muestra de átomos fríos.

El estudio de sistemas cuánticos ha sido de particular interés para la física experimental en años recientes. En particular, en los últimos 30 años ha habido un tremendo avance en el desarrollo de gases cuánticos de átomos fríos\cite{PhysRevLett.24.156,PhysRevLett.59.2631}, abriendo muchas posibilidades de realizar experimentos novedosos en física atómica. La interacción de estos medios materiales cuánticos con luz es estudiada por la \textit{óptica cuántica}.

Las interacciones átomo--fotón tienen un gran potencial como herramienta de estudio en correlaciones cuánticas y entrelazamiento\cite{Lukens:15}. En particular, la generación de pares de fotones (o \textit{bifotones}) en ensambles atómicos ha generado mucho interés para implementarlos para procesamientos en información cuántica\cite{PhysRevLett.101.190501}.

La técnica más estándar para generar bifotones es la de procesos de conversión espontánea paramétrica descendiente\cite{burnham} (o SPDC por sus siglas en inglés) en cristales no lineales, como BBO. Sin embargo, el tiempo de coherencia de los fotones generados en SPDC es muy corto (del orden de ps) gracias a que poseen un amplio ancho de banda. Esto impide realizar experimentos donde sea importante hacer mediciones con tiempos precisos; muchos sistemas de detección fotónica con tecnología de punta todavía tienen una resolución temporal de al menos decenas de ps\cite{Lukens:15}. Además, esta corta longitud de coherencia los hacen poco viables para interacciones átomo--fotón, haciéndolos poco deseables para estudios de información cuántica\cite{PhysRevLett.88.243602}.

Por esto, en años recientes ha aumentado mucho el interés en generar pares de fotones con un ancho de banda angosto. Una solución simple es colocar el cristal para SPDC en una cavidad óptica\cite{PhysRevLett.101.190501}. Sin embargo, el avance en técnicas de enfriamiento de átomos ha permitido estudiar la generación de bifotones en gases atómicos fríos por medio de procesos de mezclado de cuatro ondas espontáneo.

El mezclado de cuatro ondas (FWM por sus siglas en inglés) es un fenómeno conocido en óptica no lineal\cite{Thiel_four-wavemixing,PhysRevA.80.063809}. En este proceso, la interacción simultánea de un medio no lineal con campos eléctricos externos permite la generación de nuevos haces coherentes de luz.
En el contexto de este trabajo, una nube de átomos fríos sirve como medio no lineal para el FWM, que con dos haces láser cercanos a resonancias atómicas puede generar dos haces de fotones coherentes\cite{PhysRevA.82.043833}.

% imagen fwm con niveles

Utilizar este proceso trae varios beneficios: bifotones con un ancho de banda muy angosto, mayor eficiencia de producción y varios parámetros experimentales para poder controlar la función de onda resultante\cite{Du:08}.

En este Laboratorio, el corazón de este futuro experimento de FWM consiste en un gas de átomos de rubidio, confinados y enfriados por medio de una trampa magneto-óptica (o MOT), construida en paralelo a este trabajo. El siguiente paso natural para este experimento es el de la caracterización de los átomos fríos atrapados en la trampa. Esta caracterización se puede hacer en función de su densidad óptica: una propiedad importante que está directamente relacionada con el número de átomos atrapados. La densidad óptica es, además, particularmente importante para experimentos de FWM pues está relacionada con la tasa de bifotones generados y su tiempo de coherencia\cite{Cere:1}.

%La optimización de la nube atómica es muy importante para obtener buenos resultados; entre más fotones se generen, mejor será la estadística del experimento. Uno de los parámetros experimentales más importantes para procesos de FWM es el de la densidad óptica máxima\cite{Du:09}.

Una secuencia experimental típica para un proceso de FWM en átomos fríos tiene una ventana temporal pequeña de apenas unos cuantos microsegundos\cite{Cho2013}. Esta secuencia requiere la sincronización de muchos eventos simultáneos como abrir o cerrar obturadores de láseres, prender campos magnéticos y controlar disparadores de cámaras o fotodiodos para la captura de datos. Uno busca, además, repetir controladamente cientos de experimentos para justificar un análisis estadístico sobre los datos obtenidos. Por lo tanto, estos experimentos exigen un sistema de recolección de datos rápido, que pueda comunicarse con los intrumentos relevantes y procese a tiempo las señales generadas.

Tales experimentos están también por naturaleza en permanente desarrollo; se encuentran en un estado constante de ser actualizados, ajustados y mejorados cada vez más. Por esto, es deseable tener un programa de adquisición que sea intuitivo, y que permita un uso, comprensión y modificación sencillos por usuarios que no necesariamente sean expertos en el funcionamiento interno del programa.

Esta tesis está dedicada principalmente a desarrollar un sistema de adquisición de datos en anticipación a futuros experimentos de mezclado de cuatro ondas en átomos fríos. Sin embargo, de manera simultánea se realiza la caracterización y optimización de nuestra trampa magneto-óptica (MOT) utilizada para atrapar y enfriar átomos de rubidio. Estos dos proyectos sirven el propósito de dejar listo el sistema que se utilizará tanto para la generación como para la adquisición de fotones en el experimento. Este trabajo es una guía para el análisis e interpretación de la estadística de fotones individuales generados en el Laboratorio, así como servir de referencia para los valores óptimos de la densidad óptica de la nube atómica como función de distintos parámetros experimentales. 

Por contener dos objetivos diferentes, se decidió desarrollar cada proyecto de manera independiente en dos capítulos principales. Así, la estructura de esta tesis es como sigue:
\begin{itemize}
\item El \textbf{Capítulo 1} es el capítulo principal de este trabajo y se concentra en el estudio de correlaciones temporales para pares de fotones. Al iniciar, se discute la motivación del experimento de mezclado de cuatro ondas para generar bifotones. Luego, se presenta una introducción teórica del estudio cuantitativo de la correlación para fuentes clásicas de luz. Se introduce también un tratamiento cuántico de la correlación y se establece una clasificación de las fuentes a partir de su estadística.

Después, se describen los distintos sistemas y programas que forman parte del sistema de adquisición de datos. Se da una visión general del funcionamiento del código del programa y de cómo se comunica con los instrumentos. También se incluyen otros sistemas y circuitos que se desarrollaron para el experimento.

Finalmente, se habla de la preparación experimental para medir correlaciones en luz láser, con el propósito de probar el sistema de adquisición. Se caracterizan los fotodiodos de avalancha y se presentan los resultados del experimento.

\item La optimización de la MOT se encuentra en el \textbf{Capítulo 2}. Se presenta la motivación para hacer espectroscopía en átomos fríos y la importancia de la densidad óptica.

En la subsección de teoría, se presenta la estructura atómica de rubidio 87. Luego, se introducen los mecanismos de enfriamiento y atrapamiento de átomos que usa la MOT. Un modelo atómico sencillo de dos niveles es introducido para explicar el fenómeno de absorción en átomos, así como para definir la densidad óptica de la nube a partir de la transmisión de un haz débil de prueba.

Por último, se presenta el montaje para las medidas de densidad óptica, así como los resultados obtenidos al variar distintos parámetros experimentales. Se da un análisis cualitativo de las mediciones y se discuten las características de los espectros registrados.

\item En el \textbf{Capítulo 3} se presentan las conclusiones de este trabajo. Contiene también la visión a futuro para la continuación de este proyecto.

\item Se incluye además un \textbf{Apéndice} con una guía de usuario extensa para el sistema de adquisición de datos del experimento.

\end{itemize}