\section{Introducción}\label{introduccion}

Consideremos un sistema de dos niveles, por ejemplo la polarización de un
fotón que puede estar polarizado horizontal ($H$) o verticalmente ($V$).
Este estado puede ser representado como la siguiente función de onda

\begin{equation}
    \ket{\varphi} = c_1 \ket{H} + c_2\ket{V},
    \label{enl:1}
\end{equation}
en donde $|c_1|^2$ y $|c_2|^2$ son las probabilidades de obtener $H$ y $V$
respectivamente al llevar a cabo una medición y por lo tanto $|c_1|^2+|c_2|^2 = 1$.