\section{Espectroscopía}\label{espectroscopia}
La primer trampa magneto-óptica del Laboratorio de Átomos Fríos y Óptica Cuántica es el corazón de los experimentos en planeación para átomos fríos de rubidio. Por medio del uso de láseres desintonizados al rojo y campos magnéticos, es posible desacelerar y atrapar átomos de rubidio. Una vez atrapados, es importante caracterizar las propiedades atómicas de este ensamble; dos características de alta importancia para este laboratorio son el número de átomos atrapados, y su temperatura. Un análisis preciso de estas propiedades es necesario para realizar experimentos como el mezclado de cuatro ondas. El propósito de este segundo capítulo experimental es hacer una caracterización de la densidad óptica de la MOT del laboratorio en función de algunos parámetros experimentales, en anticipación a la generación de bifotones por FWM.
\subsection{Motivación}\label{motivacion_espectro}
El estudio de los átomos fríos ha sido un campo con mucho interés en años recientes, pues permiten preparar gases atómicos difusos para experimentos con luz.

Estudiar las propiedades de átomos a temperatura ambiente resulta complicado debido al movimiento térmico que presentan. Este movimiento --- explicado por la teoría cinética de gases --- resulta en velocidades promedio de casi 300ms$^{-1}$ para átomos de rubidio a temperatura ambiente, por ejemplo. Una velocidad esperada tan alta resulta en un ensanchamiento de las líneas espectrales debido al efecto Doppler, haciendo difícil estudiar los átomos usando técnicas espectroscópicas\cite{RevModPhys.70.721}. Es de aquí que surge el interés de enfriar átomos a temperaturas muy bajas.

El premio Nobel de Física de 1997 fue otorgado a William Phillips, Claude Cohen-Tannoudji y Steven Chu por el ``desarrollo de métodos para enfriar y atrapar átomos con luz láser", que terminaría por aplicarse en el desarrollo de la trampa magneto-óptica (MOT). En la actualidad, el uso de MOTs es estándar para el enfriamiento de átomos neutros a temperaturas del orden de cientos de $\mu$K. Estos átomos fríos son usados en una gran variedad de experimentos como interferometría atómica\cite{1742-6596-80-1-012047} o información cuántica\cite{Gross995} o son enfriados aún más para hacer experimentos en condensados de Bose-Einstein\cite{901989}.

Dos parámetros experimentales importantes de la nube atómica fría son la densidad óptica y la temperatura de los átomos atrapados. La densidad óptica es particularmente importante para experimentos de FWM, pues define distintos regímenes de conversión espontánea paramétrica descendente y determina la forma de la función de correlación $\g{2}(\tau)$ para los bifotones generados\cite{Du:08,PhysRevA.75.033814}.

En este segundo capítulo, se presenta el arreglo óptico que se usará próximamente para la generación de fotones correlacionados en FWM, así como una optimización de la densidad óptica de la MOT del laboratorio.

\subsubsection{Preparación del sistema}\label{espectro_prep}
Para realizar las medidas de la densidad óptica (OD) de nuestra nube se aprovechó el montaje óptico para la luz de bombeo del FWM (Figura \textbf{BOMBEO}). En este arreglo, se usan dos haces de bombeo co-propagantes de 780nm y 776nm (bombeo 1 y bombeo 2). Estos haces son resonantes a las transiciones \textbf{HOLA}, respectivamente. En la discusión de FWM de \cite{srivathsan} y \cite{gulati}, en esta configuración se generarán pares de fotones con modos bien definidos por medio del FWM espontáneo. Sin embargo, a los haces se les superpone un haz semilla de 795nm resonante a la transición \textbf{HOLA OTRA VEZ} para hacer más fácil la alineación y acoplamiento a fibra de la luz generada.

\begin{figure}[h!]
\centering
\includegraphics[width=0.95\linewidth]{optica_bombeo.png}
\caption{Óptica de bombeo para FWM. Un divisor de haz polarizante (PBS) permite superponer el haz semilla con el haz de bombeo 1. El filtro IF3 sirve para superponer los dos haces de bombeo.}\label{fig:optica_bombeo}
\end{figure}

El haz de prueba que se usó está en resonancia con la transición de enfriamiento (\textbf{TRANSICIÓN, correspondiente al haz de bombeo 1}) para obtener una referencia de la OD y el número de átomos en el centro de la nube. El haz de prueba pasa directamente por el centro de la nube atómica y después es analizado por un fotodiodo \textbf{THORLABS}, cuya señal se observó en un osciloscopio.

\begin{figure}[h!]
\centering
\includegraphics[width=0.95\linewidth]{optica_OD.png}
\caption{Arreglo para medir la densidad óptica.}\label{fig:optica_OD}
\end{figure}

\subsection{Teoría}\label{teoria_espectro}
%quiero hablar de la importancia de la desintonía en las mediciones, para eso necesito hablar primero de la estructura fina e hiperfina (solo un poco), de la estructura del rubidio, y efecto zeeman. luego hablo de MOT? o eso en motivación? solo un poco; hablar de láseres de rebombeo y enfriamiento (eso ya en la parte de MOT)
%
%\begin{itemize}
%\item rubidio
%\item reglas dipolares
%\item beer lambert / densidad atómica / núm. átomos
%\item espectroscopía de absorción
%\end{itemize}
Los átomos alcalinos son popularmente usados para experimentos de átomos fríos por varias razones: su estructura atómica se parece a la del hidrógeno (en que tienen un sólo electrón de valencia en su última capa) y facilita su descripción teórica\cite[p.~60]{Foot}, presiones de vaporización cercanas a temperatura ambiente\cite{steck87}, y frecuencias de excitación cercanas al visible (que permiten usar láseres comerciales).

%El alcalino usado en la MOT del Laboratorio es rubidio, que tiene dos isótopos que ocurren naturalmente con distintas abundancias: $^{85}$Rb (72.2\%) y $^{87}$Rb (27.8\%)\cite{steck87}. En este trabajo nos centramos en el isótopo de $^{87}$Rb porque es el usado para FWM, aunque el desarrollo también puede aplicarse a $^{85}$Rb.

\subsubsection{Estructura atómica}
Para poder entender el enfriamiento y confinamiento de átomos, se parte de la estructura de sus niveles energéticos\cite{steck87,robinson}. Para un átomo alcalino, su electrón de valencia presenta un momento angular orbital $\bm{L}$ y un momento angular de espín $\bm{S}$. Estos dos momentos angulares interactúan entre sí y resultan en un momento angular electrónico total $\bm{J} = \bm{L} + \bm{S}$. Por la naturaleza vectorial del acoplamiento $\bm{LS}$, este momento angular puede tener valores
\begin{equation}
|L-S| \leq J \leq |L+S|. \label{j}
\end{equation}
Esto (junto con correcciones relativistas a la ecuación de Schrödinger), es conocido como la \textit{estructura fina} e introduce un desdoblamiento de los niveles energéticos de los átomos. Es común usar la notación espectroscópica para etiquetar los niveles atómicos:
\begin{equation}
\nonumber N^{2S+1}L_J,
\end{equation}
donde $N$ es el número cuántico principal. Para el estado base de $^{87}$Rb, $L = 0$ y $S = 1/2$, por lo que $J = 1/2$, quedando el estado base como $5^2S_{1/2}$. El primer estado excitado del rubidio tiene $L = 1$, así que de la ecuación \ref{j}, $J = 1/2$ y $J = 3/2$. De este desdoblamiento surgen las transiciones \textbf{D1} ($5^2S_{1/2} \rightarrow 5^2P_{1/2}$) y \textbf{D2} ($5^2S_{1/2}\rightarrow 5^2P_{3/2}$).

Más allá de la estructura fina existe un segundo desdoblamiento de los estados energéticos llamado \textit{estructura hiperfina}. Este desdoblamiento se da por el acoplamiento del momento angular total $\bm{J}$ y el espín nuclear $\bm{I}$, que resulta en un momento angular \textit{total} $\bm{F} = \bm{J} + \bm{I}$. De manera similar a la estructura fina, el momento angular total puede valer
\begin{equation}
|J-I| \leq F \leq |J+I|.
\end{equation}
El $^{87}$Rb tiene espín nuclear $I = 3/2$, por lo que para el estado base $F = 1$ y $F=2$. Para el estado excitado, $F$ puede tener valores 0, 1, 2 y 3.

\textbf{líneas de rubidio!!}

\subsubsection{MOT}
Una descripción teórica a profundidad de los mecanismos de una MOT no forma parte de esta tesis. Sin embargo, una introducción breve basada en \cite{Foot} resulta útil.

Por un lado, el enfriamieneto láser es una manifestación de una fuerza radiativa que Ashkin llamó \textit{fuerza de scattering}\cite{PhysRevLett.24.156}. Por el otro lado, para atrapar estos átomos se aprovecha su momento dipolar magnético intrínseco con el uso de un campo magnético cuadrupolar externo\cite{PhysRevLett.59.2631}.

Este enfriamiento láser está basado en la transferencia de momento entre luz y átomos. Al absorber un fotón, el átomo sentirá una fuerza en la dirección de propagación del fotón. Además, el átomo eventualmente emitirá un fotón en una dirección aleatoria por el proceso de emisión espontánea. Al tomar el promedio estadístico para muchos fotones, la fuerza que siente el átomo por emisión espontánea será cero por no emitir en una dirección preferencial.

Esta fuerza resultante de \textit{scattering} dependerá de la diferencia de la frecuencia del láser ($\omega$), la frecuencia de resonancia atómica ($\omega_0$) y su corrimiento de frecuencia debido al efecto Doppler ($kv$) (pues los átomos están en movimiento), y será en la dirección de propagación del láser.

Un láser colimado seleccionará átomos moviéndose en una sola dirección para frenarlos. Como un átomo es libre de moverse en tres dimensiones, es necesario usar tres láseres ortogonales y contrapropagantes para reducir su velocidad neta en cada dirección. La fuerza total que sentirán los átomos en cada una de estos ejes será
\begin{align}
\nonumber F_{mel} &= F_{scat}(\omega-\omega_0-kv) - F_{scat}(\omega-\omega_0 +kv)\\
\nonumber &\simeq F_{scat}(\omega-\omega_0) - kv\frac{\partial F_{scat}}{\partial \omega} - \left[ F_{scat}(\omega-\omega_0) + kv\frac{\partial F_{scat}}{\partial\omega} \right]\\
&\simeq -2k\frac{\partial F_{scat}}{\partial \omega}v = -\alpha v. 
\end{align}

Esta fuerza se parece a la que siente una partícula en un líquido viscoso. Es por eso que esta técnica de enfriamiento láser se llamó \textit{melaza óptica}. Se puede demostrar que el coeficiente de amortiguamiento $\alpha$ es proporcional a $-(\omega - \omega_0)$. Para que esta fuerza sea amortiguante es necesario que $\alpha > 0$, \textit{i.e.} $\omega-\omega_0 <0$, que significa que una \textit{desintonía al rojo} de la frecuencia de resonancia es necesaria.

No basta enfriar átomos a temperaturas muy bajas, también es necesario lograr confinarlos a una región pequeña para poder estudiarlos en una MOT.

\subsection{Absorción}
La nube atómica es modelada como un vapor ópticamente denso para el cual un haz monocromático en resonancia puede excitar a los átomos por medio de absorción estimulada. Al re-emitir de manera aleatoria esta luz, se presenta una atenuación del haz de prueba original que permite estudiar la absorción de los átomos. 

Seguiremos el desarrollo de Foot \cite{Foot}. Supongamos que tenemos un sistema atómico de dos niveles. Si consideramos un haz pasando por un medio de ancho infinitesimal $\text{d}z$ con $N$ átomos por unidad de volumen, este medio tendrá entonces $N\text{d}z$ átomos por unidad de área. La fracción de fotones que serán absorbidos por los átomos de este medio será $N\sigma\text{d}z$, donde $\sigma$ se define como la sección transversal de los átomos. La atenuación del haz puede describirse por:
\begin{equation}
\frac{\text{d}I}{\text{d}z} = -\kappa(\omega)I \equiv -N\sigma(\omega)I, \label{beer_diff}
\end{equation}
donde $\kappa(\omega)$ es el coeficiente de absorción para la luz incidente. Esta se conoce como la \textit{Ley de Beer-Lambert} y es válida para un haz con baja intensidad que deja a la mayoría de la población en el estado base.

Se puede demostrar que para este sistema de dos niveles resonante en $\omega_0$, la sección transversal de los átomos es
\begin{equation}
\sigma(\omega) = \sigma_0 \cfrac{\Gamma^2}{4(\omega-\omega_0)^2+\Gamma^2}.
\end{equation}
Aquí, $\sigma_0$ es la sección transversal máxima para $\omega=\omega_0$, y $\Gamma$ es el \textit{ancho de banda natural} de la transición. Este ancho de banda es el ancho de la línea espectral de la transición y puede verse como la tasa de decaimiento del estado excitado.

Para intensidades más altas, la población atómica en el estado excitado crecerá y presentará emisión estimulada, que aumentará la intensidad del haz después de pasar por la nube. La transición puede entonces \textit{saturarse}: un haz intenso y en resonancia hará que el sistema oscile rápidamente entre el estado base y el estado excitado. Tomando estos efectos de saturación, se puede demostrar que el coeficiente de absorción se vuelve
\begin{align}
\nonumber \kappa(\omega,I) &= \cfrac{N\sigma(\omega)}{1+(\sigma(\omega)/\sigma_0)(I/I_{sat})}\\
& = N\sigma_0\cfrac{\Gamma^2}{4(\omega-\omega_0)^2 + \Gamma^2(1+I/I_{sat})}. \label{kappa}
\end{align}
$I_{sat} = \frac{\pi}{3}\frac{hc \ \Gamma}{\lambda^3}$ se define como la intensidad de saturación. La atenuación quedará de manera más general como $\text{d}I/\text{d}z = -\kappa(\omega,I)I$.

Si $I \ll I_{sat}$ podemos despreciar el cociente $I/I_{sat}$ y entonces regresaremos a la ecuación \ref{beer_diff}, que tiene una solución:
\begin{equation}
I(z) = I_0 \exp(-\kappa(\omega)z). \label{beer}
\end{equation}
Para una nube atómica de longitud $L$ iluminada por un haz poco intenso y frecuencia $\omega$, la intensidad que se medirá en un fotodiodo después de ser atenuada por la nube será:
\begin{equation}
I(L) \equiv I = I_0 \exp\left(-\text{OD}\cfrac{\Gamma^2}{4(\omega-\omega_0)^2 + \Gamma^2}\right).
\end{equation}
OD $=NL\sigma_0$ es la \textit{densidad óptica} de la nube y es un parámetro experimental importante para la MOT.
\newpage
\subsection{Experimento}\label{experimento_espectro}
\subsubsection{imagen}

\subsection{Resultados}\label{resultados_espectro}
