\section{Espectroscopía}\label{espectroscopia}

La primer trampa magneto-óptica del Laboratorio de Átomos Fríos y Óptica Cuántica es el corazón de los experimentos en planeación para átomos fríos de rubidio. Por medio del uso de láseres desintonizados al rojo y campos magnéticos, es posible desacelerar y atrapar átomos de rubidio. Una vez atrapados, es importante caracterizar las propiedades atómicas de este ensamble; dos características de alta importancia para este laboratorio son el número de átomos atrapados, y su temperatura. Un análisis preciso de estas propiedades es necesario para realizar experimentos como el mezclado de cuatro ondas. El propósito de este segundo capítulo experimental es hacer una caracterización de la densidad óptica de la MOT del laboratorio en función de algunos parámetros experimentales, en anticipación a la generación de bifotones por FWM.

\subsection{Motivación}\label{motivacion_espectro}

El estudio de los átomos fríos ha sido un campo con mucho interés en años recientes, pues permiten preparar gases atómicos difusos para experimentos con luz.

Estudiar las propiedades de átomos a temperatura ambiente resulta complicado debido al movimiento térmico que presentan. Este movimiento --- explicado por la teoría cinética de gases --- resulta en velocidades promedio de casi 300ms$^{-1}$ para átomos de rubidio a temperatura ambiente, por ejemplo. Una velocidad esperada tan alta resulta en un ensanchamiento de las líneas espectrales debido al efecto Doppler, haciendo difícil estudiar los átomos usando técnicas espectroscópicas\cite{RevModPhys.70.721}. Es de aquí que surge el interés de enfriar átomos a temperaturas muy bajas.

El premio Nobel de Física de 1997 fue otorgado a William Phillips, Claude Cohen-Tannoudji y Steven Chu por el ``desarrollo de métodos para enfriar y atrapar átomos con luz láser", que terminaría por aplicarse en el desarrollo de la trampa magneto-óptica (MOT).

\subsubsection{MOT}
En la actualidad, el uso de MOTs es estándar para el enfriamiento de átomos neutros a temperaturas del orden de cientos de $\mu$K. No basta enfriar átomos a temperaturas muy bajas, también es necesario lograr confinarlos a una región pequeña para poder estudiarlos.

Por un lado, el enfriamieneto láser es una manifestación de una fuerza radiativa que Ashkin llamó \textit{fuerza de scattering}\cite{PhysRevLett.24.156}. Por el otro lado, para atrapar estos átomos se aprovecha su momento dipolar magnético intrínseco con el uso de un campo magnético cuadrupolar externo\cite{PhysRevLett.59.2631}. 

\subsubsection{Preparación del sistema}
\subsubsection{Densidad óptica importancia}
\subsection{Teoría}\label{teoria_espectro}
\begin{itemize}
\item rubidio
\item reglas dipolares
\item beer lambert / densidad atómica / núm. átomos
\item espectroscopía de absorción
\end{itemize}

\newpage
\subsection{Experimento}\label{experimento_correlaciones}
\subsubsection{imagen}

\subsection{Resultados}\label{resultados_correlaciones}
