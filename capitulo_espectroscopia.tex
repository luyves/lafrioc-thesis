
\section{Problema y objetivos}\label{problema}

\subsection{Laboratorios de física contemporánea}\label{laboratorios}

Los últimos dos laboratorios de la carrera (Laboratorios de Física Contemporánea I y II que se imparten en la Facultad de Ciencias de la UNAM), son ideales para
presentar a los estudiantes experimentos que involucren conceptos de mecánica cuántica, dado que a
esa altura el material necesario ya ha sido cubierto o en su defecto el estudiante cuenta con las
bases necesarias para entender los experimentos a realizarse. Es durante estos dos laboratorios
que se llevan a cabo prácticas de física moderna, que en principio son las más apegadas al
quehacer actual de la
ciencia. Sin embargo, la finalidad de los laboratorios se ve opacada por el hecho de que el material
presente no permite realizar correctamente muchas prácticas o, en otros casos, resulta un dolor de
cabeza lidiar con los materiales disponibles. Esto debido a varias razones, el material es viejo,
los equipos que utilizan gases como argón, bario, etcétera tienen contaminaciones, las fuentes de
poder no proveen la corriente que sus medidores señalan y muestran histéresis, los espectrómetros se saturan
fácilmente, los interferómetros tienen espejos chuecos, los polarizadores y filtros están rayados,
algunos equipos ya no pueden realizar todas las mediciones que en principio podían. Y es entendible,
se necesita una gran cantidad de presupuesto para conseguir material nuevo, por un lado, y por otro
los materiales pasan por las manos de estudiantes que no siempre son tan cuidadosos como se desearía
que lo fueran. Es por esto que la modalidad de prácticas rotativas representa un programa importante
dentro de los laboratorios de física contemporánea. En estos, los estudiantes pueden ir a
laboratorios especializados en cierta área y llevar a cabo experimentos relacionados. Les es posible
tener un acercamiento más adecuado a la física experimental, y ver (superficialmente) cómo es
trabajar en un laboratorio de la UNAM. Otro punto importante es que durante las prácticas rotativas
es posible atraer el interés de los estudiantes hacia ciertas áreas de la investigación experimental
que probablemente no conocen, además de mostrar un panorama más amplio que las prácticas que se
realizan regularmente en los Laboratorios de Física Contemporánea, sin necesidad de invertir fuertes
cantidades de dinero en los laboratorios del de enseñanza con los que se cuenta en el Departamento de Física de la Facultad de Ciencias de la UNAM, ya que, básicamente, toda la
infraestructura y los materiales han sido subsidiados por la Universidad.


\subsection{Finalidad}\label{finalidad}

El presente proyecto de apoyo a la docencia busca pues, presentar a los estudiantes conceptos tales
como la no localidad de la mecánica cuántica y los estados entrelazados, teniendo en consideración
que los estados entrelazados, como se comentó anteriormente, han ido adquiriendo importancia en el
área de la información y computación cuántica, y esta rama a su vez ha ido ganando fuerza a través
de los años. Por ejemplo, se ha pasado de factorizar el número 15 con el algoritmo de Shor en 2001
\cite{Vandersypen} a factorizar 56,153 en 2012 con el algoritmo de minimización \cite{Dattani}, y
teleportar información a más de 100 km \cite{Takesue}, entre otras cosas. En estos casos el
entrelazamiento cuántico juega un papel fundamental para el funcionamiento y realización de los
experimentos. El hecho de que las desigualdades de Bell puedan ser utilizadas como una medida de
entrelazamiento del sistema es una característica importante, no sólo se necesitan estados
entrelazados para la realización del experimento, sino que además el mismo experimento da
información acerca de estos estados.

Se busca además generar en el estudiante interés en el área de la mecánica cuántica
experimental para lograr acercar a cada vez más gente a esta rama de la física, que sin duda en
años venideros dará de que hablar como ha venido sucediendo.

El artículo pues, fue concebido como un manual corto para realizar la práctica propuesta, en donde
se comenten conceptos importantes, se haga una pequeña revisión histórica y se expliquen algunos
conceptos nuevos para los estudiantes (tales como la \textit{SPDC}). Esto podría ser aprovechado en
su mayoría por cualquier universidad de habla hispana (el artículo está escrito en español dado el
tipo de público que se desea alcanzar).

Se desea que esta práctica sea implementada no solamente en la UNAM sino que otras universidades se
sumen a la idea de acercar a los estudiantes a la mecánica cuántica experimental por medio de
experimentos sencillos que estén conceptualmente a su alcance.