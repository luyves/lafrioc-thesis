\section{Correlaciones}\label{correlaciones}

El Laboratorio de Átomos Fríos y Óptica Cuántica del Instituto de Física de la UNAM --- parte del Laboratorio Nacional de Materia Cuántica --- tiene como propósito hacer investigación en metrología e información cuántica, por medio del estudio de sistemas cuánticos ópticos y materiales.

El experimento principal del Laboratorio es el de generación de pares de fotones (o \textit{bifotones}) por medio de un proceso no lineal llamado \textit{mezclado de cuatro ondas} --- o FWM por sus siglas en inglés. En el laboratorio, este proceso de FWM es realizado actualmente en una muestra de átomos de rubidio calientes, y próximamente se realizará también en una muestra de átomos fríos.
\newpage
\subsection{Motivación}\label{motivacion_correlaciones}

\subsubsection{Preparación del sistema}


\subsubsection{Instrumentos}
\newpage
\subsection{Teoría}\label{teoria_correlaciones}

\subsubsection{Óptica clásica}\label{optica_clasica}
\newpage
\subsection{Sistemas}\label{sistemas_correlaciones}

\subsubsection{Perfilómetro}\label{perfilometro}

\subsubsection{Sistema de adquisición}\label{adquisicion}

\subsubsection{Gating}\label{gating}

\newpage
\subsection{Experimento}\label{experimento_correlaciones}

\subsection{Resultados}\label{resultados_correlaciones}
