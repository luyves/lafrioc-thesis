\section*{\centering Resumen}

Esta tesis presenta el desarrollo de un sistema de adquisición de datos para pares de fotones generados por procesos de mezclado de cuatro ondas en un gas de átomos fríos. Para probar este sistema, se realizó un experimento para medir la función de correlación temporal de segundo orden $\g{2}$ para luz coherente clásica. Se obtuvo un valor de $\g{2}(0) = 1.0006\pm0.0012$, que se encuentra en excelente acuerdo con la teoría.
De manera paralela se construyó una trampa magneto-óptica (MOT) para atrapar átomos de $^{87}$Rb,y esta tesis también presenta algunos de los pasos naturales siguientes al realizar estudios espectroscópicos de los átomos atrapados con el propósito de medir su densidad óptican (OD). Se obtuvo un valor máximo de OD $= 46.9 \pm 0.4$. Finalmente, se presenta una caracterización de esta densidad óptica para distintos parámetros ajustables dentro del Laboratorio de Átomos Fríos y Óptica Cuántica del Instituto de Física de la UNAM, con el fin de servir como una referencia experimental para futuras optimizaciones.
\clearpage