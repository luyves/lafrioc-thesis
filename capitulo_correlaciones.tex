\section{Sistema de adquisición de datos}\label{correlaciones}

%El Laboratorio de Átomos Fríos y Óptica Cuántica del Instituto de Física de la UNAM --- parte del Laboratorio Nacional de Materia Cuántica --- tiene como propósito hacer investigación en metrología e información cuántica, por medio del estudio de sistemas cuánticos ópticos y materiales.
%
%El experimento principal del Laboratorio es el de generación de pares de fotones (o \textit{bifotones}) por medio de un proceso no lineal llamado \textit{mezclado de cuatro ondas} --- o FWM por sus siglas en inglés. En el Laboratorio, este proceso de FWM es realizado actualmente en una muestra de átomos de rubidio calientes, y próximamente se realizará también en una muestra de átomos fríos.

%\subsection{Motivación}\label{motivacion_correlaciones}

%La técnica más estándar para generar bifotones es la de procesos de conversión espontánea paramétrica descendiente\cite{burnham} (o SPDC por sus siglas en inglés) en cristales no lineales, como BBO. Sin embargo, el tiempo de coherencia de los fotones generados en SPDC es muy corto (del orden de ps) gracias a que poseen un amplio ancho de banda. Esto impide ciertos experimentos interesantes; incluso sistemas de detección fotónica con tecnología de punta tienen una resolución temporal de al menos decenas de ps\cite{Lukens:15}. Además, la corta longitud de coherencia los hacen poco viables para interacciones átomo--fotón, haciéndolos poco deseables para estudios de información cuántica\cite{PhysRevLett.88.243602}.
%
%Por esto, en años recientes ha aumentado mucho el interés en generar pares de fotones con un ancho de banda angosto. Una solución simple es colocar el cristal para SPDC en una cavidad óptica\cite{PhysRevLett.101.190501}. Sin embargo, el avance en técnicas de enfriamiento de átomos ha permitido estudiar la generación de bifotones en gases atómicos fríos por medio de procesos de FWM espontáneo. Utilizar este proceso trae varios beneficios: bifotones con un ancho de banda muy angosto, mayor eficiencia de producción y varios parámetros experimentales para poder controlar la función de onda resultante\cite{Du:08}. En este laboratorio, el corazón de este futuro experimento consiste en un gas de átomos de rubidio, confinados y enfriados por medio de una trampa magneto-óptica (o MOT). Una breve explicación de nuestra MOT se dará en la sección \ref{mot}, y una descripción más detallada puede encontrarse en \cite{Adrian}. A pesar de que el proceso de FWM en átomos fríos todavía no se implementa en el laboratorio, hay varios aspectos del experimento que necesitan desarrollo, como los sistemas de control y adquisición de datos así como la caracterización y optimización de la nube atómica.

%La motivación inicial para la generación de bifotones en el Laboratorio es el estudio de su correlación temporal. Esto nos da información cuantitativa acerca de su naturaleza cuántica para realizar posteriormente otros experimentos.

%Para poder analizar la correlación de bifotones, es necesario usar un sistema de adquisición capaz de analizar los pares de fotones generados por procesos de mezclado de cuatro ondas. En este capítulo se presenta una introducción a la teoría de estadística de fotones individuales y su implementación experimental, así como el desarrollo del sistema de adquisición de datos que será usado por el Laboratorio para estos experimentos en átomos fríos de $^{87}$Rb. %A continuación se presentará la motivación y desarrollo del trabajo de este capítulo. 

%\subsubsection*{Luz cuántica}\label{luzcuantica}

El primer propósito del estudio de fotones individuales es estudiar las correlaciones entre ellos. En 1986, Grangier et al\cite{grangier} generaron haces de fotones individuales utilizando decaimientos atómicos en cesio para demostrar algunas propiedades cuánticas de la luz. En particular, buscaban estudiar las correlaciones entre fotodetectores para las salidas de transmisión y reflección en un divisor de haz. Si --- citando a Grangier --- \textit{sólo se puede detectar un fotón una sola vez}, entonces habremos probado propiedades granulares de la luz y no habría duda de que sólo se puede describir de manera cuántica, i.e., con su función de onda. Así, la medición de correlaciones temporales entre fotones es una herramienta fundamental para estudiar la naturaleza cuántica de la luz.

En este capítulo se presenta una introducción a la teoría de estadística de fotones individuales y su implementación experimental, así como el desarrollo del sistema de adquisición de datos para experimentos de fotones individuales.

\subsection{Teoría}\label{teoria_correlaciones}
En esta sección se presentará una descripción de los elementos teóricos utilizados en este capítulo. Partiendo de una descripción clásica de la luz, se introducirán los conceptos de \textit{coherencia} de primer y segundo grado, que nos permitirán clasificar luz de distintas fuentes de acuerdo a la estadística que siguen. Finalmente, se realizará una cuantización del campo. Este desarrollo sigue principalmente la exposición de Kenyon\cite{Kenyon}, Loudon\cite{Loudon} y Fox\cite{Fox}.

\subsubsection{Óptica clásica}\label{optica_clasica}
Aunque muchos efectos ópticos clásicos pueden ser descritos por óptica geométrica, para poder explicar efectos como interferencia y difracción, se necesita el tratamiento clásico de la luz como ondas. La teoría de Maxwell de la luz como ondas electromagnéticas está descrita en términos del campo eléctrico $\E(\textbf{r},t)$ y el campo magnético $\B(\textbf{r},t)$; para el campo electromagnético en materiales, se definen de manera más general el campo de desplazamiento $\D(\textbf{r},t)$ y la cantidad magnética $\Hmag(\textbf{r},t)$. En un material homogéneo e isotrópico, la descripción de estos campos generales es:
\begin{align*}
\D &= \epsilon_0\epsilon_r\E & \B &= \mu_0\mu_r\Hmag,
\end{align*}
donde $\epsilon_0$ y $\mu_0$ son la permitividad eléctrica y permeabilidad magnética del vacío, respectivamente, y $\epsilon_r$ y $\mu_r$ la permitividad eléctica y permeabilidad magnética del medio.

Las ecuaciones que describen la respuesta de un medio al campo eléctrico y magnético fueron compiladas por Maxwell:
\begin{align}
\divr\D &= \rho, \\
\divr\B &= 0, \\
\curl\E &= -\frac{\partial \B}{\partial t}, \\
\curl\Hmag &= \bm{j} + \frac{\partial \D}{\partial t},
\end{align}
donde $\rho$ es la densidad de carga libre y $\bm{j}$ la corriente de carga libre. En el vacío (y sin densidad ni corriente de carga libre) $\epsilon_r = \mu_r = 1$ y las ecuaciones de Maxwell son:

\begin{align}
\divr\E &= 0, \label{gauss_elec}\\
\divr\B &= 0, \label{gauss_mag}\\
\curl\E &= -\frac{\partial \B}{\partial t}, \label{faraday}\\
\curl\B &=\mu_0\epsilon_0\frac{\partial \E}{\partial t}, \label{ampere}
\end{align}

Tomando el rotacional de la ecuación \ref{faraday}, y combinándolo con la ecuación \ref{ampere}:
\begin{equation*}
\curl(\curl\E) = -\frac{\partial}{\partial t}\curl\B = - \mu_0\epsilon_0\frac{\partial}{\partial t}\frac{\partial\E}{\partial t} = -\mu_0\epsilon_0\frac{\partial^2\E}{\partial^2 t}.
\end{equation*}
Por otro lado, utilizando que para cualquier campo vectorial $\bm{\mathcal{A}}$ es cierto que:
\begin{equation}
\nonumber \curl(\curl \bm{\mathcal{A}}) = \nabla(\divr \bm{\mathcal{A}}) - \nabla^2\bm{\mathcal{A}},
\end{equation}
y el hecho que el primer término del lado derecho es cero por la ecuación \ref{gauss_elec}, tenemos que
\begin{equation}
\nabla^2\E(\bm{r},t) = \mu_0\epsilon_0\frac{\partial^2\E(\bm{r},t)}{\partial^2 t},
\end{equation}
que corresponde a una ecuación de onda con velocidad $c = 1/\sqrt{\mu_0\epsilon_0} = 2.998\times 10^8$ ms$^{-1}$, que es la velocidad de la luz en el vacío.
De una manera análoga se encuentra una ecuación de onda para el campo magnético $\B$.

\subsubsection{La función de correlación de primer orden}\label{coherencia}
Para poder describir la correlación temporal entre dos haces de luz, es necesario entender el efecto de interferencia. La interferencia es el efecto de la recombinación de dos ondas en movimiento; patrones de interferencia ocurren cuando estas dos ondas presentan una diferencia relativa de fase. Un patrón de interferencia es indispensable para estudiar la coherencia entre ondas. %En el contexto del campo electromagnético, existen varios ejemplos que sirven para dejarlo en claro: el experimento de la doble rendija de Young, el interferómetro de Michaelson o el interferómetro de Mach Zehnder.

Si dos trenes de onda presentan una relación bien determinada de fase entre ellos, se dice que son \textit{coherentes}. En una situación ideal, si conocemos esta relación de fase para un tiempo inicial, podríamos deducir la fase relativa en todo momento. La vida real, por desgracia, no funciona así y en realidad tenemos fluctuaciones en la fase que suceden conforme pasa el tiempo. Sin embargo, si la diferencia de fase entre las dos ondas permanece relativamente constante dentro de un intervalo $\tau_c$, decimos que son \textit{parcialmente coherentes} con un tiempo de coherencia $\tau_c$. En este sentido, la coherencia es una medida de la estabilidad de frecuencia de la luz: podremos predecir acertadamente la fase de una onda para un tiempo $t + \tau$, siempre y cuando $\tau << \tau_c$. De manera inmediata podemos obtener la distancia de coherencia $d_c = c\tau_c$ y medir la fase en dos puntos distintos en el espacio.

\begin{figure}[]
\centering
\includegraphics[width=0.65\linewidth]{mach_zehnder.png}
\caption[Interferómetro de Mach-Zehnder.]{Interferómetro de Mach-Zehnder. El haz original $\mathcal{E}$ se divide en dos brazos con distancia $l_1$ y $l_2$.}\label{fig:mach_zehnder}
\end{figure}


Para cuantificar la coherencia, calculamos la \textit{correlación}. La Figura \ref{fig:mach_zehnder} muestra un esquema sencillo del interferómetro de Mach Zehnder. Supongamos que tenemos un campo eléctrico $\E(t)$ incidente sobre el primer divisor de haz, que asumimos idéntico al segundo y ambos con coeficiente de reflexión $\mathcal{R}$ y coeficiente de transmisión  $\mathcal{T}$. Después de pasar por el primer divisor, el haz reflejado y el transmitido recorren una distancia $l_1$ y $l_2$, respectivamente, usualmente diseñadas de distinta magnitud. Al recombinarse en el segundo divisor, tendremos dos salidas:
\begin{align*}
\E_A(t) &= \mathcal{RT}\E(t_1) + \mathcal{TR}\E(t_2) \propto \E(t_1) + \E(t_2),\\
\E_B(t) &= \mathcal{TT}\E(t_2) + \mathcal{RR}\E(t_1),
\end{align*}
con $t_1 = t - l_1/c$ y $t_2 = t - l_2/c$.

Consideremos la intensidad del campo $\E_A(t)$ al tiempo $t$:
\begin{align*}
I_A(t) &\propto |\E(t_1) + \E(t_2)|^2 = \mathcal{E}(t_1)^2 + \mathcal{E}(t_2)^2 + 2\E(t_1)\cdot\E(t_2).
\end{align*}
En la realidad no podemos medir la intensidad de manera instantántea, por lo que tomamos la intensidad promediada:
\begin{equation}
\langle I_A(t)\rangle \propto \langle\mathcal{E}(t_1)^2\rangle + \langle\mathcal{E}(t_2)^2\rangle + 2\langle\E(t_1)\cdot\E(t_2)\rangle.
\end{equation}

Vemos que tenemos tres contribuciones a la intensidad medida. Las primeras dos corresponden a la intensidad que mediríamos de cada campo después de recorrer su brazo del interferómetro de manera independiente, \textit{i.e.}, sin efectos de interferencia. Sin embargo, el tercer término contiene la información de la \textit{correlación} de cada haz.

Si --- sin pérdida de generalidad --- consideramos que $t_1 < t_2 = t_1 + \tau$ (y $\tau > 0$), podemos ver que la correlación dependerá de esta diferencia temporal $\tau$ a partir de un tiempo dado, más que de valores particulares de $t_1$ o $t_2$, siempre y cuando la naturaleza de $\E(t)$ nos permita tomar promedios sin mucho problema. De manera más general, el tercer término está definido de manera estadística como:
\begin{equation}
\langle\E(t)\cdot\E(t+\tau)\rangle = \cfrac{1}{T}\int_T dt \ \E^*(t)\E(t+\tau)\equiv \langle\E^*(t)\E(t+\tau)\rangle ,
\end{equation} 

que se conoce como la \textit{función de correlación de primer orden}. El \textit{grado de coherencia temporal de primer orden} se define como la versión normalizada de la función de correlación de primer orden:
\begin{equation}
\g{1}(\tau) = \cfrac{\left\langle \mathcal{E}^*(t)\mathcal{E}(t+\tau) \right\rangle}{\left\langle\mathcal{E}^*(t)\mathcal{E}(t)\right\rangle}.
\end{equation}  

Veamos cómo se ve $\g{1}(\tau)$ para distintos tipos de luz.
\subsubsection*{Luz caótica}\label{luz_termica}

Consideremos a una fuente de luz caótica como un ensamble de muchos átomos que emiten luz de manera independiente unos de otros. Si se da una colisión entre ellos, la fase de la luz emitida cambia aleatoriamente (que permanece constante hasta que ocurra otra colisión) pero su amplitud $E_0$ y frecuencia $\omega_0$ permanecen constantes. Podemos modelar este cambio de fase como una función $\varphi(t)$ con dominio $[0,2\pi]$, y si observamos este campo en un punto espacial fijo:
\begin{equation}
\mathcal{E}(t) = E_0e^{-i\omega_0t + i\varphi(t)}.
\end{equation}
El cambio de fase de la luz emitida por un átomo es independiente de todos los demás. Suponiendo que podemos superponer cada campo individual para obtener el campo eléctrico total:
\begin{align*}
\mathcal{E}(t) &= \E_1(t) + \E_2(t) + ... + \E_n(t)\\
\mathcal{E}(t) &= E_0e^{-i\omega_0t}\left(e^{i\varphi_1(t) + i\varphi_2(t) + ... + i\varphi_n(t)}\right)
\end{align*}

Si calculamos $\g{1}(\tau)$ para esta fuente de luz:

\begin{align*}
\langle\mathcal{E}^*(t)\mathcal{E}(t+\tau)\rangle = E_0^2 e^{-i\omega_0\tau}&\langle\{e^{-i\varphi_1(t)} + e^{-i\varphi_2(t)} + ...  +e^{-i\varphi_n(t)}\}\\
&\times\{e^{i\varphi_1(t+\tau)} + e^{i\varphi_2(t+\tau)} + ...  +e^{i\varphi_n(t+\tau)}\}\rangle.
\end{align*}
Al tomarse el promedio estadístico, los términos cruzados se eliminan pues corresponden a saltos aleatorios distintos (y promedian cero al considerarse el ensamble completo). Así,
\begin{align}
\nonumber\langle\mathcal{E}^*(t)\mathcal{E}(t+\tau)\rangle &=  E_0^2e^{-i\omega_0\tau}\sum_{j=1}^n\langle e^{i\phi_j(t+\tau) - i\phi_j(t)}\rangle\\
\nonumber&= nE_0^2e^{-i\omega_0\tau}\langle e^{i\phi_j(t+\tau) - i\phi_j(t)}\rangle\\
&= n\langle\mathcal{E}_j^*(t)\mathcal{E}_j(t+\tau)\rangle, \label{eq15}
\end{align}
ya que cada átomo es indistinguible de los demás.

Calculemos $\langle\mathcal{E}_j^*(t)\mathcal{E}_j(t+\tau)\rangle$. De la teoría cinética de los gases, sabemos que la probabilidad de que haya un tiempo entre colisiones dentro del intervalo $[\tau,\tau+d\tau]$ (donde la fase del campo es constante) es:
\begin{equation}
p(\tau)d\tau = (1/\tau_c)e^{(-\tau/\tau_c)}d\tau,
\end{equation}
donde $\tau_c$ es el tiempo característico de vuelo, o tiempo de coherencia. Usando esta distribución de probabilidad para calcular $\langle\mathcal{E}_j^*(t)\mathcal{E}_j(t+\tau)\rangle$,
\begin{align}
\nonumber \langle\mathcal{E}_j^*(t)\mathcal{E}_j(t+\tau)\rangle &= E_0^2e^{-i\omega_0\tau}\langle e^{i\phi_j(t+\tau) - i\phi_j(t)}\rangle\\
\nonumber &= E_0^2e^{-i\omega_0\tau}\int_\tau ^\infty p(\tau') d\tau'\\
\nonumber &= E_0^2e^{-i\omega_0\tau} e^{-\tau/\tau_c}.
\end{align}

Finalmente, de la ecuación \ref{eq15} obtenemos que:
\begin{equation}
\g{1}(\tau) = e^{-i\omega_0\tau- \tau/\tau_c}. \label{eq_colision}
\end{equation}

Para una fuente de luz caótica con ensanchamiento Doppler se puede hacer un análisis similar. Partiendo de la expresión del campo eléctrico para este tipo de luz:
\begin{equation*}
\E(t) = E_0\sum_{i=1}^n e^{-i\omega_i t + i\varphi_i}.
\end{equation*}

Aunque la fase del $i$-ésimo átomo es constante (dado que asumimos que no hay colisiones atómicas), están distribuidos de manera aleatoria y las contribuciones de fase de distintos átomos promediará cero. Además, cada átomo tiene una frecuencia de radiación $\omega_i$ recorrida de la frecuencia central $\omega_0$, determinada por su velocidad. 

Calculando $\langle\mathcal{E}^*(t)\mathcal{E}(t+\tau)\rangle$, obtenemos que:
\begin{equation}
\g{1}(\tau) = e^{-i\omega_0\tau - \frac{\pi}{2}(\tau/\tau_c)^2}. \label{eq_doppler}
\end{equation}

De las ecuaciones \ref{eq_colision} y \ref{eq_doppler}, vemos que la magnitud de $\g{1}(\tau)$ para luz caótica ensanchada por colisiones tiene un perfil Lorentziano, mientras que para luz caótica ensanchada por Doppler presenta un perfil Gaussiano. En ambos casos, $|\g{1}(0)| = 1$ es el máximo y decrece monótonamente hacia $0$.

\subsubsection*{Luz láser}\label{luz_laser}

De manera ideal, un láser emite una sola frecuencia y tiene coherencia perfecta. Podemos describirlo por una onda sinusoidal con una fase $\varphi$ bien definida en todo momento, y observándolo en un punto fijo a lo largo de la dirección de propagación:
\begin{equation}
\E(t) = E_0 e^{-i\omega_0t + i\varphi}. \label{eq19}
\end{equation}

Se sigue entonces que:
\begin{align*}
\langle\mathcal{E}^*(t)\mathcal{E}(t+\tau)\rangle &= E_0^2 e^{-i\omega_0\tau},
\end{align*}
y
\begin{equation}
\g{1}(\tau) = e^{-i\omega_0\tau}.
\end{equation}

Es inmediato que $|\g{1}(\tau)| = 1$ para todo $\tau$, que significa que la luz láser es perfectamente coherente. Sin embargo, un láser real emite en un intervalo de frecuencias centrado en $\omega$ y típicamente presenta tiempos de coherencia de milisegundos\cite{coherence}.

%\begin{figure}[h!]
%\centering
%\includegraphics[width=0.9\linewidth]{g1_all.png}
%\caption{$|\g{1}(\tau)|$ para luz láser, luz caótica ensanchada por colisiones y luz caótica ensanchada por effecto Doppler.}\label{fig:g1}
%\end{figure}

\subsubsection{La función de correlación de segundo orden}
El experimento de interferencia de intensidades que realizaron Hanbury Brown y Twiss\cite{hbt} en 1956 fue uno de los primeros estudios que mostró que el análisis de las correlaciones entre intensidades era interesante. Justo como consideramos el grado de coherencia temporal de primer orden como una medida de la correlación entre amplitudes del campo eléctrico, el \textit{grado de coherencia temporal de segundo orden} es una medida de la correlación entre sus intensidades. La función $\g{2}(\tau)$ mide esencialmente fluctuaciones en la intensidad y se define como:
\begin{equation}
\g{2}(\tau) = \cfrac{\langle\mathcal{E}^*(t)\mathcal{E}^*(t+\tau)\mathcal{E}(t+\tau)\mathcal{E}(t)\rangle}{\langle\mathcal{E}^*(t)\mathcal{E}(t)\rangle \langle\mathcal{E}^*(t+\tau)\mathcal{E}(t+\tau)\rangle} = \cfrac{\langle I(t)I(t+\tau)\rangle}{\langle I(t)\rangle\langle I(t+\tau)\rangle}.
\end{equation}

De acuerdo a esta descripción\cite{Loudon}, es posible probar con la desigualdad de \textit{Cauchy-Schwartz} que:
\begin{align}
1 &\leq \g{2}(0) \label{g0_clasica}\\
\g{2}(\tau) &\leq \g{2}(0). \label{g0_gt_clasica}
\end{align}

Para una fuente clásica de luz láser como la descrita por la ecuación \ref{eq19}, es trivial probar que:
\begin{equation}
\g{2}(\tau) = 1 \label{g2_laser_teoria}
\end{equation}
para todo $\tau$. La luz láser también es coherente temporalmente en intensidad.

Para las fuentes de luz caótica se puede demostrar que, si se tiene un número $n$ muy grande de átomos, es válido:
\begin{equation}
\langle\mathcal{E}^*(t)\mathcal{E}^*(t+\tau)\mathcal{E}(t+\tau)\mathcal{E}(t)\rangle = n^2\left[\langle\mathcal{E}_i^*(t)\mathcal{E}_i(t)\rangle^2 + |\langle\mathcal{E}_i^*(t)\mathcal{E}_i(t+\tau)\rangle|^2\right]. \label{eq22}
\end{equation}

Usando la ecuación \ref{eq22} en la definición de $\g{2}$, tenemos que para luz caótica (y para $n\gg1$):
\begin{equation}
\nonumber \g{2}(\tau) = 1+|\g{1}(\tau)|^2,
\end{equation}
quedando para luz caótica con colisiones y con ensanchamiento Doppler:
\begin{align}
\g{2}(\tau) &= 1 + e^{-2\tau/\tau_c}, & \g{2}(\tau) &= 1 + e^{-\pi(\tau/\tau_c)^2}.
\end{align}
En la Figura \ref{fig:g2} vemos la forma de $\g{2}(\tau)$ para las tres fuentes clásicas de luz tratadas hasta ahora. Notamos que en los tres casos, las ecuaciones \ref{g0_clasica} y \ref{g0_gt_clasica} se cumplen.
\begin{figure}[h!]
\centering
\includegraphics[width=0.9\linewidth]{g2_all.png}
\caption{$\g{2}(\tau)$ clásica para luz láser, luz caótica ensanchada por colisiones y luz caótica ensanchada por effecto Doppler.}\label{fig:g2}
\end{figure}

\subsubsection{Cuantización del campo}\label{optica_cuantica}
Hasta ahora, sólo hemos tratado de forma clásica las fuentes de luz. Nos gustaría ver qué pasa con la función $\g{2}(\tau)$ al seguir un tratamiento cuántico. De la teoría clásica del campo electromagnético \cite{Loudon}, sabemos que el campo eléctrico $\E(\bm{r},t)$ y el campo magnético $\B(\bm{r},t)$ son campos vectoriales dependientes del tiempo que en vacío dependen de un tercer campo $\A(\bm{r},t)$ (el potencial vectorial) y un campo escalar $\phi(\bm{r},t)$:
\begin{align*}
\B(\bm{r},t) &=\curl\A(\bm{r},t), \\
\E(\bm{r},t) &=-\nabla\phi(\bm{r},t) - \cfrac{\partial\A(\bm{r},t)}{\partial t}.
\end{align*}

De acuerdo a las expresiones anteriores, el Hamiltoniano clásico tiene la forma
\begin{equation}
H = \frac{1}{2}\epsilon_0 \int\int\int_V d^3r \ \left(|\E(\bm{r},t)|^2 + c^2|\B(\bm{r},t)|^2\right). \label{Hamiltoniano}
\end{equation}

Tomando la norma de Coulomb, (para la cual $\divr\A = 0$), se puede demostrar que la expansión de Fourier del potencial vectorial es:
\begin{equation}
\A(\bm{r},t) = \sum_{\bm{k}}\sum_{\mu =\pm 1}\left(\bm{e}^{(\mu)}(\bm{k})a_{\bm{k}}^{(\mu)}(t)e^{i\bm{k}\cdot\bm{r}} + \bm{e}^{*(\mu)}(\bm{k})a^{*(\mu)}_{\bm{k}}(t)e^{-i\bm{k}\cdot\bm{r}}\right),
\end{equation}
con $\bm{e}^{(\pm 1)}$ los vectores unitarios de polarización circular y $\bm{k}$ es el vector de onda.

Si aplicamos las reglas de cuantización que postula Loudon\cite{Loudon} para los coeficientes de la serie de Fourier $a_{\bm{k}}^{(\pm 1)}(t)$, es posible demostrar que el Hamiltoniano dado se vuelve:
\begin{equation}
\hat H = \sum_{\bm{k},\mu} \hbar\omega_{\bm{k}}\left(\hat a^{\dagger(\mu)}(\bm{k})\hat a^{(\mu)}(\bm{k}) + \frac{1}{2}\right).
\end{equation}

Este Hamiltoniano se asemeja mucho al de un oscilador armónico expresado en función de los operadores $\ad$ de creación y $\hat a$ de aniquilación:
\begin{equation}
\hat H = \hbar\omega\left(\ad\hat a + \frac{1}{2}\right).
\end{equation} 
Esto, y la naturaleza ondulatoria y periódica de la luz, hacen razonable describir el comportamiento cuántico en el formalismo del oscilador armónico. Los eigenestados de este Hamiltoniano son $|n\rangle$, y les corresponde una energía:
\begin{align}
E_n &= \hbar\omega\left(n + \frac{1}{2}\right) & n &= 1,2,3,... \label{eq_energia_oscilador}
\end{align}

Los operadores $\ad$ y $\hat a$ actúan en estos estados de la forma:
\begin{align*}
\ad\hat a|n\rangle &= \hat n|n\rangle,\\
\hat a|n\rangle &= \sqrt{n}|n-1\rangle,\\
\ad|n\rangle &= \sqrt{n+1}|n+1\rangle.
\end{align*}

En la descripción de óptica cuántica, estos estados $|n\rangle$ describe el número de excitaciones con energía $\hbar\omega$; estas excitaciones reciben el nombre de \textit{fotones}.

\subsubsection{Estadística de fotones individuales}\label{estadistica}
La óptica cuántica está encargada de estudiar a los haces de luz como un chorro de fotones en vez de la descripción clásica de ondas. Para hacer esto, es útil estudiar la estadística de este haz de fotones. La intensidad de este campo está dada por el valor esperado del operador $\hat n$, que es equivalente al número promedio de fotones en el haz. Esto quiere decir que:
\begin{equation}
\langle I \rangle \propto \langle \hat n \rangle  = \langle \ad \hat a \rangle \equiv \bar n.
\end{equation}
Con esto, vale la pena volver a definir la función de correlación de segundo orden en función del operador $\hat n$:
\begin{align}
\g{2}(\tau) = \cfrac{\langle \ad(t)\ad(t+\tau) \hat a(t+\tau)\hat a(t)\rangle}{\langle\ad(t)\hat a(t)\rangle\langle\ad(t+\tau)\hat a(t+\tau)\rangle} = \cfrac{\langle \hat n(t) \hat n (t+\tau)\rangle}{\langle \hat n(t)\rangle \langle \hat n(t+\tau)\rangle}. \label{g2_cuantica}
\end{align}

Veamos los resultados de $\g{2}(\tau)$ para las fuentes de luz usadas en la sección \ref{coherencia}. Para un haz coherente, nos preguntamos cuál es la probabilidad $\mathcal{P}(n)$ de encontrar $n$ fotones en un haz de longitud $L$ partido en $N$ subsegmentos. Esto resulta equivalente a encontrar $n$ subsegmentos con un sólo fotón y $N-n$ con ninguno, en cualquier orden. Este problema se puede describir con la distribución binomial:
\begin{equation}
\nonumber\mathcal{P}(n)  = \cfrac{N!}{n!(N-n)!}p^n(1-p)^{N-n}.
\end{equation}

Como $p = \bar n/N$ es la probabilidad de encontrar un fotón en un subsegmento, tenemos que:
\begin{equation}
\nonumber \mathcal{P}(n) = \cfrac{N!}{n!(N-n)!}\left(\cfrac{\bar n}{N}\right)^n\left(1-\cfrac{\bar n}{N}\right)^{N-n} = \cfrac{1}{n!}\left(\cfrac{N!}{(N-n)!N^n}\right)\bar n ^n\left(1-\cfrac{\bar n}{N}\right)^{N-n}.
\end{equation}

Queremos tomar el límite cuando $N\rightarrow\infty$. Usando la fórmula de Stirling y el teorema del binomio para demostrar que:
\begin{align*}
&\lim_{N\rightarrow\infty}\left[\cfrac{N!}{(N-n)!N^n}\right] = 1, & &\left(1-\cfrac{\bar n}{N}\right)^{N-n} = \exp(-\bar n),
\end{align*}
y tenemos que la probabilidad $\mathcal{P}(n)$ cuando $N\rightarrow\infty$ es:
\begin{align}
\mathcal{P}(n) &= \cfrac{\bar n^n}{n!}e^{-\bar n}, & n&=0,1,2,... \ ,
\end{align}
que es una \textbf{distribución Poissoniana}. En general, una distribución Poissoniana expresa la probabilidad de que ocurra un número de eventos en un tiempo dado si hay una frecuencia de ocurrencia media y cada evento es independiente de todos los demás. Esto significa que la detección de cada fotón está aleatoriamente espaciada. Para este tipo de distribuciones, la \textit{varianza} está dada por:
\begin{equation}
(\Delta n)^2 =\sum_{n=0}^\infty(n-\bar n)^2\mathcal{P}(n) = \bar n, \label{varianza}
\end{equation}
lo que significa que la \textit{desviación estándar} cumple con
\begin{equation}
\Delta n = \sqrt{\bar n}.
\end{equation}

Por otro lado, la luz térmica está definida como la radiación electromagnética emitida por un cuerpo negro, y presentan una densidad de energía dentro del rango de emisión $[\omega, \omega + d\omega]$ de acuerdo a la Ley de Planck:
\begin{equation}
\nonumber\rho(\omega,T)d\omega = \frac{\hbar\omega^3}{\pi^2c^3}\cfrac{d\omega}{\exp(\hbar\omega/k_BT)-1}.
\end{equation}
Cada modo oscilatorio tiene la energía dada por la ecuación \ref{eq_energia_oscilador} y, de acuerdo a la mecánica estadística, la probabilidad de que se encuentren $n$ átomos en el modo correspondiente a $\omega$ es:
\begin{equation}
\mathcal{P}_{\omega}(n) = \cfrac{\exp(-n\hbar\omega/k_BT)}{\sum_{n=0}^\infty\exp(-n\hbar	\omega/k_BT}). \label{prob_termica}
\end{equation}
Para este tipo de luz, es posible demostrar que $\bar n = 1/(\exp(\hbar\omega/k_BT)-1)$, y la probabilidad \ref{prob_termica} es:
\begin{equation}
\mathcal{P}_{\omega}(n) = \cfrac{1}{\bar n+1}\left(\cfrac{\bar n}{\bar n + 1}\right)^n,
\end{equation}
que es una \textbf{distribución de Bose-Einstein}. Para fuentes de luz con esta distribución,
\begin{equation}
(\Delta n)^2 = \bar n + \bar n^2 \geq \bar n.
\end{equation}
Este resultado nos muestra que la varianza de una distribución de Bose-Einstein es mayor que la de una distribución Poissoniana. Las fluctuaciones de intensidad en la luz caótica clásica se asemejan mucho a las de la luz térmica descritas aquí, y presentan una estadistica similar.

Con este análisis, es conveniente definir una clasificación para las estadísticas de fotones en función de su desviación estándar:
\begin{itemize}
\item \textbf{Super-Poissoniana}: $\Delta n > \sqrt n$,
\item \textbf{Poissoniana}: $\Delta n = \sqrt n$,
\item \textbf{Sub-Poissoniana}: $\Delta n < \sqrt n$. 
\end{itemize}

La interpretación de que la luz térmica sea super-Poissoniana significa que los fotones que se detectan vienen en promedio con una separación temporal menor a la frecuencia media de ocurrencia, lo que sugiere que llegan en ``paquetes" (o \textit{bonches}). A este efecto se le conoce como \textit{bunching}.

¿Qué pasa cuando se toman medidas de intensidad al mismo tiempo? Para estudiar ests medidas en coincidencia, veamos a $\g{2}(\tau = 0)$. Utilizando la regla de conmutación $[\hat a,\ad] = 1$ podemos encontrar de la ecuación \ref{g2_cuantica} que:
\begin{equation}
\g{2}(0) = \cfrac{\langle\hat n^2\rangle - \langle\hat n\rangle}{\langle\hat n\rangle} = 1+ \cfrac{\Delta n^2 - \bar n}{\bar n ^2}.
\end{equation}
Vemos con esto que para luz perfectamente coherente, $\g{2}(0) = 1$ y que para luz térmica, $\g{2}(0) > 1$. Sin embargo, si tuviéramos una distribución sub-Poissoniana, se tiene que $\g{2}(0) < 1$. Notemos que en la descripción clásica, de acuerdo a la ecuación \ref{g0_clasica}, cualquier campo electromagnético cumple que $\g{2}(0) \geq 1$ y no es posible encontrar un equivalente clásico al resultado para una distribución sub-Poissoniana.

Este resultado ---contrastando con el efecto de bunching--- significa que los fotones tienden a no llegar al detector muy cerca uno de otro. Si este efecto es constante (es decir, si los fotones llegan regularmente espaciados uno después de otro), se le conoce como \textit{antibunching}.\footnote{Es importante decir que una estadística sub-Poissoniana no necesariamente quiere decir que hay detrás un proceso óptico que presente antibunching\cite{PhysRevA.41.475}. Sin embargo, por simplicidad, se le considerará así en este trabajo pues son dos efectos que muy frecuentemente se presentan juntos.} El hecho que $\g{2}(0) < 1$ nos dice que la fuente del campo electromagnético que se está analizando no puede ser descrito de manera clásica y es un claro ejemplo de la naturaleza cuántica de la luz.

Podemos ver los resultados obtenidos en esta sección para la estadística de fotones en la Tabla \ref{valores_g0}:
\begin{table}[h!]
\centering
\begin{tabular}{|c|c|c|c|}
\hline
\textbf{Descripción clásica} & \textbf{Efecto de fotones} & \textbf{Estadística} & \textbf{$\g{2}(0)$} \\ \hline
Luz caótica o térmica        & Bunching                   & Super-Poissoniana          & $>1$                \\ \hline
Coherente                    & Aleatorios                 & Poissoniana		           & $=1$                \\ \hline
---                          & Antibunching               & Sub-Poissoniana            & $<1$                \\ \hline
\end{tabular}
\caption{$\g{2}(0)$ y su correspondiente estadística y descripción clásica para distintas fuentes de luz.}\label{valores_g0}
\end{table}

\subsection{Sistemas}\label{sistemas_correlaciones}
La principal parte del trabajo de este capítulo consistió en la elaboración de distintos programas y sistemas que serán de utilidad para el experimento de FWM en átomos fríos. Esto consistió en el sistema óptico de captura, en un perfilador de haces Gaussianos, la paquetería del sistema de adquisición, y la electrónica de \textit{gating}.

\subsubsection{Sistema óptico de captura}\label{bombeo}

El arreglo óptico se divide en la parte de bombeo (la luz que llega a la nube de átomos) y la parte de adquisición (la luz generada por el FWM que será capturada). Esta sección se enfoca en el sistema óptico de adquisición. La Figura \ref{fig:optica_fwm} muestra este arreglo óptico. Se sigue la notación usada en la Figura \ref{fig:fwm_lineas} con fines didácticos, pues todavía no se introducen las transiciones atómicas relevantes a este proceso de FWM. %Como todavía no se introducen los niveles atómicos que participan en este FWM, 

\begin{figure}[h!]
\centering
\includegraphics[width=0.95\linewidth]{optica_adq_fwm1.png}
\caption[Óptica de adquisición para FWM.]{Óptica de adquisición para FWM. El filtro IF1 deja pasar la luz $f3$ y el IF2 deja pasar la luz $f4$. Un color más tenue indica que la intensidad de los haces generados es muy baja. Esta luz se enviará al TDC para analizar correlaciones.}\label{fig:optica_fwm}
\end{figure}

En este montaje experimental, los haces de bombeo son colineales y los fotones generados también lo serán. El montaje de la óptica de adquisición consiste principalmente en un arreglo de filtros de interferencia SEMROCK LL01-780 y LL01-808 para filtrar la luz que se generará en el FWM de los haces de atrapamiento que son mucho más potentes. A cada ángulo, estos filtros de interferencia transmiten una longitud de onda central y reflejan el resto. Después de filtrar los haces de bombeo, la luz generada fue acoplada a dos fotodiodos de avalancha (APDs) para la detección de fotones individuales por medio de fibras monomodo. Como se muestra en la Figura \ref{fig:optica_fwm}, para la detección y procesamiento de señales se utilizan dos instrumentos: algunos APD de silicón \textit{id120} y un etiquetador de cuentas temporales \textit{id800} (o TDC) de IDQuantique.

Un APD es un un diodo p-n con una ganancia alta. Estos instrumentos tienen un mecanismo interno de amplificación, que genera señales eléctricas con alta tasa señal-ruido como respuesta a detecciones de un sólo fotón\cite{Cork}. Las cuentas detectadas por un APD pueden tener distintas fuentes:
\begin{itemize}
\item Fotones individuales, que pueden ser registrados correctamente o como \textit{afterpulses}.
\item \textit{Cuentas obscuras}, señales generadas por el mismo APD sin luz externa incidente. La tasa de cuentas obscuras es dependiente de la temperatura del detector y el voltaje de polarización (bias). El fabricante del \textit{id120} indica un valor de cuentas obscuras de $<200$ Hz para un voltaje de polarización máximo.
\end{itemize}

Los APD \textit{id120} cuentan con una zona de detección de 500 $\mu$m de diámetro y una eficiencia cuántica de 80\% para $800$ nm. Es necesario optimizar los valores del voltaje de polarización aplicados para minimizar el \textit{afterpulsing} y la tasa de cuentas obscuras. Una descripción general más detallada de el funcionamiento y optimización de APDs puede encontrarse en \cite{Michalet}. Una desventaja de la elección de \textit{id120} es que no tienen un trigger interno capaz de ser activado de manera externa, por lo que fue necesario diseñar un circuito AND (\textit{gating}) utilizando búferes de tres estados. Este circuito se construyó como parte de este trabajo y su descripción se encuentra en la sección \ref{gating}.

Finalmente, el \textit{id800} es un módulo etiquetador de cuentas temporales (time-to-digital converter, o TDC por sus siglas en inglés). Cuenta con 8 canales de entrada BNC con una resolución temporal de 81 ps para coincidencias de eventos en canales distintos. El \textit{id800} cuenta con un software de LabView para el procesamiento de cuentas, pero la funcionalidad que ofrecía no era adecuada para experimentos de FWM. Por ello, se optó por escribir una biblioteca completa para Python 3.6 para permitir la programación del instrumento. Con esta biblioteca, se escribió un programa para procesar y visualizar en tiempo real las cuentas que el \textit{id800} reciba.

Aún con técnicas estándar hoy en día, la tasa de generación de bifotones en experimentos de este tipo es baja\cite{Cere:1}; es por esto que es importante obtener la mayor eficiencia posible a la hora de acoplar las fibras ópticas. El problema del acoplamiento consiste en la alineación del haz incidente y de la fibra para maximizar la potencia transferida. 
%En condiciones ideales, uno podría conseguir un porcentaje de acoplamiento a la fibra de casi 100\%\cite{Ladany:93}, pero hacerlo es laborioso y necesita de múltiples elementos ópticos. 
El acoplamiento de luz láser a una fibra monomodo es un problema de empatamiento de modos\cite{Newport}. Para un láser con distribución Gaussiana, el diámetro ($1/e^2$) $D$ incidente sobre una lente de distancia focal $f$ necesario para producir una mancha focalizada de diámetro $\omega$ es:
\begin{equation}
\label{acoplamiento}
f = D\frac{\pi \omega}{4\lambda},
\end{equation}

donde $\lambda$ es la longitud de onda del láser. La eficiencia del acoplamiento dependerá de qué tanto podemos empatar el tamaño de la mancha focalizada $\omega$ con el diámetro de la fibra óptica. De la ecuación \ref{acoplamiento}, vemos que el diámetro del haz es un parámetro controlable importante para conseguir un buen acoplamiento. Por esto, se decidió escribir un sencillo programa de computadora para analizar perfiles de haces láser utilizando fotos tomadas con un chip CCD. Este programa se usó para controlar el tamaño de los haces que serán acoplados a las fibras que van a los APDs. La descripción del perfilómetro de láseres se encuentra en la siguiente sección.

\subsubsection{Perfilómetro láser}\label{perfilometro}
%Poder caracterizar y manipular un haz láser de manera adecuada siempre es ventajoso en un laboratorio de óptica pues permite optimizar algunas aplicaciones de dicha luz, como acoplamiento a fibras ópticas.

Una manera de caracterizar un haz es analizando su perfil de intensidad espacial en un plano perpendicular a su dirección de propagación. La propagación de los láseres del laboratorio puede aproximarse bastante bien asumiendo que tienen una distribución Gaussiana en 2-D, que corresponde a un perfil de intensidad:
\begin{equation}
I(r) = I_0 e^{\frac{-2r^2}{w^2}},
\end{equation}
donde $r^2 = x^2 + y^2$ y $w$ representa el diámetro del haz. $w$ es en realidad una función de $w(z)$ a partir de la distancia en el que el frente de onda es plano; para un corte transversal es constante \cite{CVI}.

\begin{figure}[h!]
\centering
\includegraphics[width=0.75\linewidth]{gaussian_waist.png}
\caption{Perfil de intensidad Gaussiano y distintas definiciones de diámetro.}\label{fig:beam_waist}
\end{figure}

Hay varias definicones de diámetro de haz, y aunque para haces Gaussianos la más común es el diámetro para el cual la intensidad ha caído a $1/e^2$ (13.5 \%) de su intensidad original. Otras definiciones --- que pueden verse en la Figura \ref{fig:beam_waist} --- son el diámetro $1/e$, \textit{full-width half-maximum} (FWHM) y $D4\sigma$, que corresponden a caídas de intensidad a $1/e$ y $50\%$ para las primeras dos, respectivamente, y 4 veces la desviación estándar para la última. Para haces Gaussianos ideales, los diámetros de $1/e^2$ y $D4\sigma$ coinciden. 

Haciendo uso de una cámara digital, es posible tomar una foto del perfil de intensidad espacial en un plano perpendicular y realizar un análisis en la computadora. La cámara disponible para ese propósito en el Laboratorio es una cámara Thorlabs DCC1545M, que cuenta con un chip CCD con una resolución de 1280x1024 pixeles cuadrados de 5.4 $\mu$m de lado. El chip CCD tiene una intensidad de saturación de aproximadamente 1 $\mu$W/cm$^2$. Los láseres que se usan en el Laboratorio no pueden producir un haz estable con intensidades tan bajas, lo que significa que atenuar la luz láser usando filtros de densidad neutral (ND) es necesario.

El programa del perfilómetro se encuentra en \cite{github:p}; sin embargo, una pequeña descripción del algoritmo se presenta aquí. Una vez que una foto ha sido tomada, es procesada por computadora y convertida en una matriz de valores dentro del intervalo $[0,255]$ (pues la cámara usada toma fotos en 8-bits). Después, el algoritmo realiza un procesamiento sencillo para reducir el ruido en la imagen que puede provenir tanto de ruido electrónico o luz de fondo. El algoritmo calcula un valor de offset basado en el valor mediano de los pixeles no iluminados por el haz.

Para el análisis de imágenes, primero hace falta encontrar el centro del haz. Sin embargo, a veces es difícil tener una imagen limpia. A veces la óptica usada está sucia o induce efectos de interferencia por una mala alineación. Por esto, encontrar el centro de un perfil de intensidad a veces no es tan directo como encontrar el pixel de mayor intensidad. Normalmente, este ruido puede arreglarse si uno usa un iris o limpia bien la óptica, pero a veces es riesgoso cuando el arreglo es grande o complicado. En casos como este, es posible hacer un preprocesamiento de la imagen para intentar limpiar un poco el ruido.

%https://www.ncbi.nlm.nih.gov/pmc/articles/PMC3749244/pdf/1472-6807-11-7.pdf
%http://www.globalsino.com/EM/page1705.html
%https://www.google.com.mx/search?client=opera&q=fourier+transform+image+mask&sourceid=opera&ie=UTF-8&oe=UTF-8
\begin{figure}[h!]
\centering
\includegraphics[width=0.8\linewidth]{fft_flowchart.png}
\caption{Diagrama de pre-procesamiento para imágenes usando la transformada rápida de Fourier.}\label{fig:fft_flow}
\end{figure}

La idea de este pre-procesamiento (Figura \ref{fig:fft_flow}) consiste en remover el ruido en el dominio de frecuencias por medio de la transformada discreta de Fourier en 2D (utilizando el algoritmo de la transformada rápida, o FFT). Para analizar espectros, es conveniente utilizar el hecho de que la transformada de una distribución Gaussiana es también una Gaussiana:
\begin{align*}
\mathcal{F}\{I(x,y)\} &= \hat I (u,v) = \int\int I_0\exp\left(-\cfrac{x^2+y^2}{r_0^2}\right)\exp(-i2\pi (ux+vy))dxdy \\
&= I_0\int \exp\left(-\cfrac{x^2}{r_0^2}\right)\exp(-i2\pi ux)dx\int\exp\left(-\cfrac{y^2}{r_0^2}\right)\exp(-i2\pi vy)dy \\
&= I_0\left(\cfrac{\sqrt{\pi}}{r_0}\exp(-\pi^2 r_0^2u^2)\right)\left(\cfrac{\sqrt{\pi}}{r_0}\exp(-\pi^2 r_0^2v^2)\right) \\
&= I_0\cfrac{\sqrt{\pi}}{r_0}\exp(-\pi^2 r_0^2(u^2+v^2)) = \hat{I_0}\exp\left(-\cfrac{u^2+v^2}{z_0^2}\right),
\end{align*}

donde $\hat I_0 = I_0\cfrac{\sqrt{\pi}}{r_0}$ es la nueva amplitud de la distribución y $z_0 = \cfrac{1}{\pi r_0}$ es el nuevo radio. Observamos que si tenemos un haz con perfil ancho, se convertirá en un perfil delgado en el espacio de frecuencias.

\begin{figure}[h!]
\centering
\includegraphics[width=\linewidth]{fft.png}
\caption[Foto real de un haz y su espectro de frecuencias al aplicarle FFT.]{Foto real de un haz y su espectro de frecuencias al aplicarle FFT. En vez de ser perfectamente Gaussiano, vemos que hay franjas de interferencia y otros defectos (ver círculos).}\label{fig:fft}
\end{figure}
\newpage
Viendo el espectro de frecuencias, podemos bloquear los puntos que no correspondan al espectro de nuestra Gaussiana. A esto lo llamaremos una \textit{máscara} o filtro. La calidad de la imagen que obtengamos dependerá de la manera en la que se filtran las frecuencias no deseadas\cite{Chen}. Finalmente, se aplica la transformada inversa de Fourier y se observa que el perfil puede haber mejorado.

\begin{figure}[h!]
\centering
\includegraphics[width=\linewidth]{fft_beams.png}
\caption{Antes y después de aplicar pre-procesamiento a la foto del haz.}\label{fig:fft_beams}
\end{figure}

Dependiendo de cuánto se bloquee en el espacio de frecuencias cambiará la calidad de la imagen reconstruida: si se bloquea muy poco, no habrá mejora, si se bloquea mucho, se puede perder información del perfil. Esta herramienta es útil cuando se prefiere una estimación medianamente precisa del ancho de un haz con mucho ruido y no se puede arreglar la imagen del con óptica.

Después del pre-procesamiento, se encuentra el \textit{centro de gravedad de la imagen} con un cálculo del primer momento de la intensidad del haz sobre la superficie del CCD en ambas direcciones\cite{ISO11146-1}:
\begin{align*}
\langle x \rangle &= \frac{\int I(x,y)x \ dxdy}{\int I(x,y) \ dxdy} & \langle y \rangle &= \frac{\int I(x,y)y \ dxdy}{\int I(x,y) \ dxdy}.
\end{align*}

El algoritmo realiza esta integral de manera discreta a lo largo de cada columna y cada fila del arreglo para encontrar el valor de la proyección sobre cada eje. Una vez encontrado el centro del haz, el algoritmo procede a hacer un ajuste a una distribución Gaussiana por medio de mínimos cuadrados.

%\begin{equation*}
%G(x) = Ae^{-\frac{(x-\mu)^2}{2\sigma^2}}.
%\end{equation*}
%https://www.researchgate.net/publication/252062037_A_Simple_Algorithm_for_Fitting_a_Gaussian_Function_DSP_Tips_and_Tricks

Además de existir como una biblioteca de funciones independiente, se escribió una interfaz gráfica para utilizar el programa como un ejecutable que se encuentra en \cite{github:p}. Actualmente el programa es utilizado por todos los usuarios del Laboratorio.

\subsubsection{Sistema de adquisición de datos}\label{adquisicion}

El sistema de adquisición consiste en la paquetería de programación del \textit{id800} y el programa de visualización de cuentas, desarrollados en este trabajo para la adquisición de datos en experimentos de bifotones. El \textit{id800} es un convertidor de cuentas de tiempo a valores digitales que consta en un circuito integrado de aplicación específica (ASIC) que registra eventos en los 8 canales del módulo y los manda a una matriz de puertas programable (FPGA) que los ordena y comprime. En la Figura \ref{fig:optica_fwm}, el \textit{id800} recibe señales eléctricas que recibe de los APDs en alguno de sus canales. El \textit{id800} cuenta con un búfer de entrada que trabaja a una velocidad de 200 millones de eventos por segundo, que resulta en una resolución temporal de 5.5 ns para eventos consecutivos registrados en el mismo canal. Sin embargo, la resolución temporal para eventos registrados en distintos canales es de 81 ps.

Para realizar la comunicación entre el \textit{id800} y el resto de los sistemas del laboratorio, fue necesario escribir una paquetería de uso. El diagrama de flujo para el desarrollo de la paquetería se encuentra en la Figura \ref{fig:flowchart}. Una descripción secuencial de lo que hace el programa es:
\begin{enumerate}
\item Establecer conexión con el \textit{id800}.
\item De manera simultánea, configurar los parámetros del experimento (que incluyen un \textit{búfer de datos} que establece el tamaño de los archivos a crear) y empieza a ``escuchar" para registrar eventos que lleguen al \textit{id800}.
\item Visualizar en tiempo real las cuentas procesadas.
\item Guardar automáticamente los eventos una vez que el búfer de datos se llene.
\end{enumerate}

\begin{figure}[]
\centering
\includegraphics[width=\linewidth]{TDC_flow.png}
\caption[Diagrama de flujo del programa de adquisición.]{Diagrama de flujo del programa de adquisición. Ver detalles en la sección \ref{adquisicion}.}\label{fig:flowchart}
\end{figure}

La paquetería de uso para el módulo etiquetador de cuentas temporales (TDC) \textit{id800} está escrita en Python 3.6. Esta paquetería (llamada \code{hunahpy}) hace fuerte uso de \code{ctypes}, que es una biblioteca de funciones foráneas para Python. Esta paquetería provee compatibilidad de estructuras con C/C++ y el uso de librerías compartidas (DLLs).

El módulo \textit{id800} venía con un software de control proprietario, un software de LabView y ejecutables compilados en C. Sin embargo, al usar el programa provisto por el fabricante limita el tipo de experimentos que se pueden realizar con el \textit{id800}, pues uno no es libre de ajustar a detalle el funcionamiento del instrumento a las necesidades del laboratorio. Por esto, el desarrollar un software de aquisición propio era importante. Utilizando los archivos de compilado en C, se escribió en Python una paquetería para controlar el \textit{id800} desde la computadora. Un manual completo se puede encontrar en el Anexo A y la paquetería está disponible en \cite{github} para uso público. Se escribió también un programa de visualización de datos en tiempo real con dos funciones específicas de conteo de datos:
\begin{itemize}
\item Una función que integra el número de eventos por unidad de tiempo para cada canal.
\item Un analizador de diferencias temporales: tomando un evento en un canal como un \textit{inicio}, y el siguiente como un \textit{fin}. De estas diferencias se construye un histograma útil para ver las diferencias temporales entre distinas señales (distintos canales). 
\end{itemize}

El programa de adquisición es parte de un sistema más completo que consiste en los sistemas de contol, de imagen y de adquisición de datos.

\subsubsection{Circuito AND}\label{gating}
Los APDs utilizados no cuentan con un canal de entrada digital que permita prenderlos o apagarlos usando un pulso electrónico. Por esto, se diseñó un pequeño instrumento que funcionara como compuerta digital para los eventos que registra el TDC. Esto permite una sincronización con el sistema de control y minimiza la cantidad de datos basura generados por las cuentas obscuras de los APDs.
%\begin{figure}[h!]
%\centering
%\includegraphics[width=0.6\linewidth]{gating_board_wht.png}
%\caption{Circuito de gating.}\label{fig:gating_board}
%\end{figure}

Esto es básicamente un circuito AND adecuado a las especificaciones de voltaje de salida de los APDs (TTL de bajo voltaje o LVTTL), de entrada del TDC (TTL o LVTTL) y los pulsos del sistema de control (TTL). Se decidió usar un búfer de tres estados SN74LVC126A de Texas Instruments, que permite 3 salidas: $1$ V, $0$ V, y un estado de alta impedancia (donde no pasa corriente). El búfer tiene dos canales de entrada (input y control) y uno de salida (output); cuando no hay señal de control, el estado del búfer es de alta impedancia. Esto fue para evitar la posibilidad de que el TDC interpretara señales de salida de bajo voltaje (idealmente $0$ V) como falsos eventos. El instrumento permite hasta 8 señales de salida independientes, con sus respectivos canales de entrada y control.

\begin{figure}[h!]
\centering
\includegraphics[width=\linewidth]{gating_plot.png}
\caption[Ejemplo de prueba para el circuito de gating.]{Ejemplo de prueba para el circuito de gating. a) El pulso de entrada o la señal que queremos usar. b) El pulso de control. c) El pulso de salida después de pasar por el circuito.}\label{fig:gating_plot}
\end{figure}
Se realizaron pruebas para verificar que el circuito funcionara adecuadamente. Con un generador de funciones, se enviaron dos trenes de pulsos TTL de distintas frecuencias: una frecuencia baja como control y una más alta como input. El pulso de menor frecuencia se usó para simular el pulso que emitirá el sistema de control, mientras que el tren de pulsos de alta frecuencia simuló las señales emitidas por los APDs. Como se observa en la Figura \ref{fig:gating_plot}, la señal de salida está regulada por el pulso de control, con lo que nos aseguramos que el \textit{id800} no recibirá datos fuera de esta ventana experimental.

\newpage
\subsection{Configuración experimental}\label{experimento_correlaciones}
Como preparación para los experimentos de FWM en átomos fríos, se diseñó un experimento sencillo para medir la función de correlación de segundo orden $\g{2}$ para luz láser.

Se amplió el montaje de la óptica de adquisición de FWM de la Figura \ref{fig:optica_fwm} para analizar la luz del haz de bombeo de 780 nm ($f1$). Se montó un interferómetro de intensidades con un divisor de haz 50/50 y cada brazo se mandó a un APD. Para este experimento se apagó el segundo haz ($f2$). La Figura \ref{fig:optica_780} muestra esta ampliación al montaje de la óptica de adquisición de FWM presentada en la sección \ref{bombeo}.

\begin{figure}[h!]
\centering
\includegraphics[width=0.95\linewidth]{optica_adq_7801.png}
\caption[Óptica de adquisición para el haz de 780 nm.]{Óptica de adquisición para el haz de 780 nm. La sección resaltada es la ampliación del arreglo óptico de la Figura \ref{fig:optica_fwm}. Se consiguió una potencia de alrededor de 3 pW usando filtros ND y un divisor no polarizante BS, necesaria para no saturar a los APDs.}\label{fig:optica_780}
\end{figure}

En la sección \ref{teoria_correlaciones} vimos que en el tratamiento cuántico podemos considerar a un haz de luz como un haz de fotones individuales. El \textit{flujo de fotones} $\Phi$ se define como el número de fotones promedio que pasa por una región por unidad de tiempo. Si consideramos un haz monocromático y con intensidad constante $I$, entonces el flujo se da como:
\begin{equation}
\Phi = \frac{IA}{\hbar\omega} = \frac{P}{\hbar\omega}\text{fotones s}^{-1}, \label{flujo}
\end{equation}
siendo $P$ la potencia del haz.

Cada APD tiene un valor de \textit{eficiencia cuántica} $\eta$, que es la proporción del número de foto detecciones al número de fotones incidentes. Para el \textit{id120} y luz de 780 nm, $\eta \sim 80\%$. Así, la tasa de fotones detectados por los APDs es:
\begin{equation}
\mathcal{R} = \eta\Phi.
\end{equation}

Existe un valor máximo de $\mathcal{R}$ para cada APD que surge de que el detector necesita ``descansar" después de registrar un evento. Al detectar un fotón, se genera una avalancha para multiplicar la señal. Sin embargo, por los altos voltajes de polarización necesarios (de hasta cientos de volts), se aplica un circuito de atenuamiento que reduce el voltaje del APD por un tiempo corto en donde no se registran nuevos eventos\cite{Cova:96}. Para el \textit{id120}, este llamado \textit{tiempo muerto} es de $\sim$1 $\mu$s. Con esto, la tasa máxima $\mathcal{R}$ de eventos que podemos detectar sin pérdidas significativas es inversa a este tiempo muerto, o $\sim 10^{6} \ \text{fotones s}^{-1}$. Con este valor máximo de $\mathcal{R}$ y $\lambda = 780$ nm (sabiendo que $\omega = 2\pi c/\lambda$) encontramos el valor máximo de la potencia del haz láser que podemos usar $P_{max} \sim 3$ pW. Esto lo conseguimos usando un arreglo de filtros de densidad neutral (ND) para atenuar la potencia original del haz láser; estos filtros reducen la potencia de un haz en $10^{\text{-OD}}$, donde OD es la densidad óptica total del arreglo de filtros.

Se realizaron pruebas para medir la tasa de cuentas obscuras para los dos APDs que se usaron en este experimento. Estas medidas se tomaron sin luz láser y se registraron los eventos por media hora usando el programa de adquisición de datos. La estadística de estas cuentas obscuras sigue una estadística Poissoniana \cite{darkcountstats}, y las tasas de cuentas obscuras registradas concuerdan con el valor máximo de 200 Hz reportado por el fabricante. La Figura \ref{fig:dark_counts} presenta un histograma de las cuentas obscuras de los dos APDs y su respectivo ajuste a una distribución Poissoniana. El tamaño de cada canasta en el histograma fue de 150 ms.

\begin{figure}[ht!]
\centering
\includegraphics[width=\linewidth]{cuentas_obscuras_2.png}
\caption[Histograma de cuentas obscuras para los APDs]{Histograma de cuentas obscuras para el APD1 (azul) y el APD2 (rojo). El tiempo de integración para cada canasta fue de 150 ms.}\label{fig:dark_counts}
\end{figure}

La temperatura de operación de los APDs fue de -40 $^{\circ}$C. Se eligió esta temperatura porque es la temperatura a la cual el fabricante reporta para la calibración de los instrumentos. El voltaje de polarización fue de 200 V para ambos APDs.  La tasa de cuentas obscuras que se obtuvo fue de:
\begin{itemize}
\item APD1: $\mu = 180.67\pm 33.55$ cuentas s$^{-1}$
\item APD2: $\mu = 155.55\pm 31.66$ cuentas s$^{-1}$
\end{itemize}

Después de probar la funcionalidad del programa de adquisición, se realizaron medidas de intensidad para luz láser de 780 nm. Para un haz estable de 100 $\mu$W de potencia, se usó un arreglo de filtros ND con densidad óptica total de 7.5 para obtener 3 pW a la entrada del divisor de haz 50/50. Cada brazo del interferómetro se acopló a una fibra mono-modal conectada a un APD. Para diferenciar cada brazo del interferómetro, a uno se le asignó el subíndice $T$ por ser el transmitido por el divisor de haz, mientras el otro tendrá $R$ por ser el reflejado.

Se ajustó la distancia de los dos brazos del interferómetro para que tuvieran la misma longitud para medir $\g{2}(\tau)$ en coincidencia ($\tau=0$). Sin embargo, esto no es muy importante. Si los brazos del interferómetro no fueran iguales, al calcular $\g{2}(\tau)$ veríamos las cuentas en coincidencia a un tiempo $\tau \neq 0$. %Esto se hizo con una incertidumbre de un par de centímetros, que  coresponde a una incertidumbre de apenas una fracción de nanosegundo. Como la escala temporal usada en este experimento fue de varias decenas de nanosegundos, por lo que para fines prácticos la diferencia temporal en la detección de cada brazo fue cero.

Al utilizar los APDs no se mide directamente la intensidad de los láseres, así que es necesario relacionarlo a cantidades que podamos medir en el laboratorio. En la práctica, la intensidad de un haz no es completamente constante y habrán fluctuaciones que se verán reflejadas en la estadística que hagamos al calcular $\g{2}(0)$. Es por esto que es adecuado considerar las probabilidades de detección para un tiempo arbitrario de integración $\Delta t$\cite{Beck}:
\begin{equation}
\g{2}(\tau = 0) = \cfrac{P_{TR}(\Delta t)}{P_T(\Delta t)P_R(\Delta t)}, \label{g2 exp}
\end{equation}
donde $P_{TR}$ es la probabilidad conjunta de medir una detección tanto en el APD $T$ (transmitido) como en el APD $R$ (reflejado) en el mismo intervalo $\Delta t$. Estas probabilidades están dadas por:
\begin{align*}
P_T(\Delta t) &= \mathcal{R}_T\Delta t,  & P_{TR}(\Delta t) &= \mathcal{R}_{TR}\Delta t & P_R(\Delta t) &= \mathcal{R}_R\Delta t, 
\end{align*} 
para las tasas de detección promedio $\mathcal{R}_T$, $\mathcal{R}_R$ y la tasa de detección conjunta $\mathcal{R}_{TR}$.

Para un tiempo total de detección $\Delta T$ --- subdividido en muchas canastas de tamaño $\Delta t$ --- podemos hacer un histograma de $n$ canastas para obtener la distribución del número de eventos registrados en cada canal:
\begin{align*}
&_{\downarrow}^{t_0} \\
T = \ &\mid 1 \mid 0 \mid 2 \mid 1\mid 0\mid 0\mid 1\mid 1 \mid 0\mid \ldots \mid 0 \mid\\
&\xleftrightarrow[\Delta t]{}\\
R = \ &\mid 0 \mid 0 \mid 1 \mid 1\mid 1\mid 0\mid 2\mid 0 \mid 0\mid \ldots \mid 1 \mid\\
\end{align*}
donde $T_i$ y $R_i$ representan el número de cuentas en $[t_i,t_{i+1}] = [i\Delta t, (i+1)\Delta t]$, por ejemplo.

Las tasas de detección promedio serán:
\begin{align*}
\mathcal{R}_T &= \left(\frac{N_T}{\Delta T}\right), & \mathcal{R}_{TR} &= \left(\frac{N_{TR}}{\Delta T}\right), & \mathcal{R}_R &= \left(\frac{N_R}{\Delta T}\right),
\end{align*}
donde $N_T$ y $N_R$ representan el número total de cuentas para cada canal. En términos de los histogramas de eventos, se ve que $N_T = \sum_{i=0}^n T_i$ y $N_R = \sum_{i=0}^n R_i$. Para $\Delta t$ pequeños, las probabilidades de detección serán también muy pequeñas y será válido que el número de cuentas conjunto sea $N_{TR} = \sum_{i=0}^n T_i R_i$.
La ecuación \ref{g2 exp} queda entonces dada por:
\begin{align}
\nonumber \g{2}(\tau = 0) &= \cfrac{N_{TR}}{N_T N_R}\left(\cfrac{\Delta T}{\Delta t}\right),\\
 &= \cfrac{\sum_i T_i R_i}{\sum_i T_i \sum_i R_i}\left(\cfrac{\Delta T}{\Delta t}\right). \label{g2_exp_1}
\end{align}
Este análisis es válido para medir $\g{2}(0)$. El retraso $\tau$ surge de la diferencia de camino óptico entre los brazos del interferómetro: uno podría modificar este retraso en tiempos al mover los acopladores de fibra a los APDs o utilizando cables de distinta longitud que vayan de los APDs al TDC, pero podemos introducir un \textit{retraso virtual} en alguno de los brazos del interferómetro en el análisis de datos para $\g{2}(\tau)$. Esto es posible porque las propiedades estadísticas de la luz láser son estacionarias --- \textit{i.e.}, sus fluctuaciones provienen de un proceso \textit{ergódico}\cite[p.~93]{Loudon}. Si desplazamos el inicio de uno de los histogramas por una casilla, simularemos que los eventos de ese brazo del interferómetro estarán retrasados por un valor de $\tau = \Delta t$.
\begin{align*}
&\ _{\downarrow}^{t_0} \\
T = \ &\mid 1 \mid 0 \mid 2 \mid 1\mid 0\mid 0 \mid 1 \mid \ldots \mid 1 \mid 0 \mid 0 \mid\\
R' &= \ \ \mid 0 \mid 0 \mid 1 \mid 1\mid 1\mid 0 \mid \ldots \mid 2 \mid 0 \mid 0 \mid 1 \mid\\
& \ \ \ \ \ ^{\uparrow}_{t'_0 = t_0 + \Delta t}
\end{align*}
Vemos que, sin embargo, sólo podremos obtener el número de cuentas conjunto $N_{TR}$ para el traslape temporal de nuestros nuevos histogramas, y nuestro tiempo total de detección $\Delta T$ también habrá cambiado. En este caso,
\begin{align*}
N_T &= \sum_{i=1}^{n} T_i, & N_R &= \sum_{i=0}^{n-1} R_i, & N_{TR} = \sum_{i=0}^{n-1} T_{i+1}R_i.
\end{align*}

Podemos generalizar este retraso para $k$ desplazamientos y denotar a $\tau_k = k\Delta t$. Finalmente, la ecuación \ref{g2_exp_1} quedará de forma más general como:
\begin{equation}
\g{2}(\tau_k) = \cfrac{\sum_i T_{i+k} R_i}{\sum_i T_i \sum_i R_i}\left(\cfrac{\Delta T_k}{\Delta t}\right), \label{g2_final}
\end{equation}
donde cada índice $i$ corre dentro de su respectivo rango. Esto nos permite calcular $\g{2}(\tau)$ para múltiplos de $\Delta t$, aunque a medida que $k$ aumente, la región de traslape de los histogramas disminuirá. Como cada traslape reduce el número de datos que podemos analizar, la precisión del cálculo de $\g{2}(\tau_k)$ también disminuirá.
\subsection{Resultados}\label{resultados_correlaciones}

El objetivo principal de este capítulo fue el de realizar mediciones con el interferómetro para luz láser y así verificar el buen funcionamiento del sistema de adquisición de datos. Con este sistema, se fijó un tamaño de búfer de 1,000,000 de eventos para el \textit{id800} y se guardaron 70 millones de cuentas para los dos APDs $T$ y $R$ por 40 minutos.

Se realizó el análisis de $\g{2}$ para cada canal y se eligió un tiempo de integración de $\Delta t =$ 2.5 ns (definido en la ecuación \ref{g2 exp}). El tiempo total de detección $\Delta T$ se calculó a partir de la primera y la última etiqueta de tiempo registrada por cada millón de eventos. Para cada valor de $\tau_k$ desde 2.5 ns hasta 125 ns se realizó un análisis estadístico sobre todos los datos obtenidos para calcular $\g{2}(\tau_k)$ de acuerdo a la ecuación \ref{g2_final}.

\begin{figure}[]
\centering
\includegraphics[width=\linewidth]{g2_laser.png}
\caption[$\g{2}(\tau)$ para un haz de luz coherente.]{$\g{2}(\tau)$ para un haz de luz coherente. La línea punteada es el resultado teórico de \ref{g2_laser_teoria}, y el cálculo de $\g{2}$ se hizo de acuerdo a \ref{g2_final}.}\label{fig:g2_result}
\end{figure}

La Figura \ref{fig:g2_result} presenta los resultados de estas mediciones. Se reporta un valor en coincidencia de $\g{2}(0) = 1.0006\pm0.0012$. Como la teoría predice un que $\g{2}(\tau) = 1$ para todo $\tau$, se reporta un promedio de $\overline{\g{2}} = 1.0007\pm0.0014$, que está en buena concordancia con lo predicho. Las incertidumbres presentadas son exclusivamente estadísticas y no se consideraron otras posibles fuentes de error. Este resultado responde bien al propósito de este trabajo: el desarrollo de un esquema experimental en anticipación a experimentos de FWM en átomos fríos. 