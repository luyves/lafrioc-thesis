\section{Paquetería de uso de \code{hunahpy}}\label{apendice_TDC}
La paquetería de \code{hunahpy} provee métodos útiles para el uso del módulo etiquetador de cuentas temporales \textit{id800}, fabricado por IDQuantique.

El nombre de \code{hunahpy} proviene de la cosmología maya narrada en el Popol Vuh: los dioses gemelos Hunahpú e Ixbalanqué, que fueron al inframundo por retar a los señores del Xibalbá y al emerger victoriosos se convirtieron en el Sol y la Luna.

La última finalidad de esta paquetería es ser utilizada para analizar correlaciones entre fotones gemelos, y el nombre \code{hunahpy} hace alusión a las semejanzas a la mitología de los dioses gemelos y su ascención a seres de luz. Eso y que la otra alternativa, \code{pyxbalanque}, sonaba muy fastidiosa.

\subsection{Introducción}
La biblioteca de uso del TDC consiste en varios métodos que aprovechan la funcionalidad del etiquetador para procesar señales de entrada. Hace uso de 3 bibliotecas compartidas:
\begin{itemize}
\item \code{libusb} y \code{nhconnect}, que proveen funcionalidad para reconocer y realizar conexiones a controladores USB.
\item \code{tdcbase}, librería compartida compilada en C por el fabricante del TDC que se comunica con el FPGA del etiquetador.
\end{itemize}

\subsection{Instalación}
Debido a que la biblioteca \code{tdcbase} está compilada en 32 bits, una distribución de 32 bits de Python es necesaria para poder correr el código.
Todos los archivos necesarios se encuentran en \textbf{github}. Una vez descargados, \textbf{FATLA SETUP PY}

\subsection{Uso}
Al importar \code{hunahpy}, se importa sólo la clase \code{TDC}, que contiene todos los métodos necesarios que se pueden llamar fácilmente:

\begin{verbatim}
############################
#from hunahpy import TDC   #
#tagger = TDC()            #
#...                       #
#tagger.getLastTimestamps()#
#tagger.close()            #
############################
\end{verbatim}

Este objeto \code{TDC} realiza automáticamente la conexión al primer módulo etiquetador que se encuentre conectado a la computadora y realiza la inicialización necesaria a partir del archivo de configuración \code{config.py}. Este archivo contiene la información del experimento como:
\begin{itemize}
\item El formato de nombre de los archivos a guardarse automáticamente,
\item El número de archivos a crearse,
\item El número de eventos a registrarse en cada archivo,
\item Los canales habilitados del TDC,
\item Los parámetros para el histograma de diferencias de tiempos en el programa de interfaz gráfica.
\end{itemize}

Estos parámetros iniciales de configuración pueden modificarse al cambiar las variables del archivo y sirven para determinar tanto la visualización de los datos entrantes en el programa gráfico como la estructura de los \textit{datafiles} creados, que pueden ser tanto en formato binario como hexadecimal. Un ejemplo del formato binario de los archivos creados se puede ver en la tabla \ref{datafile}.
\begin{table}[]
\label{datafile}
\centering
\begin{tabular}{|c|c|}
\hline
\textbf{Etiqueta} & \textbf{Canal} \\ \hline
340891716         & 0              \\ \hline
352001146         & 0              \\ \hline
353852682         & 1              \\ \hline
355704280         & 1              \\ \hline
357556058         & 0              \\ \hline
359407737         & 1              \\ \hline
361259559         & 0              \\ \hline
371221842         & 0              \\ \hline
372371107         & 0              \\ \hline
374223109         & 1              \\ \hline
376075177         & 0              \\ \hline
377927291         & 1              \\ \hline
379779439         & 0              \\ \hline
390892503         & 0              \\ \hline
392744469         & 1              \\ \hline
...				  & ...
\end{tabular}
\caption{Archivo binario de datos creado.}
\end{table}

Para la opción de output binario, los valores de las etiquetas de tiempo están dados en múltiplos del reloj (\code{timebase}) del sistema que es aproximadamente $81ps$, que puede obtenerse precisamente usando el método de \code{timebase} para los objetos TDC.

\subsubsection{Lista de métodos}
El objeto \code{TDC} tiene una lista de propiedades que determinan los parámetros de operación. Como referencia, se listarán los métodos de acuerdo a la siguiente leyenda, con un ejemplo:
\begin{itemize}
\item output(y precisión, si la tiene) \code{método(argumentos)}: Descripción del método. Si un método no tiene argumentos o no regresa un output, su descripción omitirá esa sección.
\item int32 \code{timestamp\_count}: El tamaño del búfer de eventos.
\end{itemize}

Las propiedades modificables del TDC son:
\begin{itemize}
\item double \code{timebase}: La resolución temporal del TDC en segundos. El valor exacto es $8.09552722121028e^{-11}s$.
\item int32 \code{timestamp\_count}: El tamaño del búfer de eventos. El búfer es el número total de eventos independientes que el TDC puede procesar en su memoria interna; si este valor está fijado en $1000$ y llega un evento $1001$, reescribirá al primero evento y así continuará. Entre $0$ y $1,000,000$.
\item int32 \code{channels\_enabled}: La \textit{máscara de bytes} de los canales habilitados del TDC. Si sólo se habilita el canal 1 (correspondiente a una máscara 1), todos los eventos que lleguen a un canal que no sea el 1 no serán procesados. Se recomienda dejar todos los canales habilitados.

La máscara de bytes es una correspondencia byte-canal que maneja el TDC. La Tabla \ref{mascara} muestra los primeros valores de la relación de correspondencia. Se recomienda dejar todos los canales habilitados (que equivale a un byte de 255, -1, o 0xff en hexadecimal), a menos que un defecto físico en un canal de entrada al TDC esté ocasionando ruido. En ese caso, es simple seguir la secuencia de máscara para encontrar el valor necesario de byte.

\begin{table}[] \label{mascara}
\centering
\begin{tabular}{|c|c|c|c|c|c|}
\hline
Byte	 & Canal   & Byte  & Canal   & Byte  & Canal \\ \hline
1    & 1       & 6    & 2,3     & 11   & 1,2,4   \\ \hline
2    & 2       & 7    & 1,2,3   & 12   & 3,4     \\ \hline
3    & 1,2     & 8    & 4       & 13   & 1,3,4   \\ \hline
4    & 3       & 9    & 1,4     & 14   & 2,3,4   \\ \hline
5    & 1,3     & 10   & 2,4     & 15   & 1,2,3,4 \\ \hline
\end{tabular}
\caption{Máscara de bytes del TDC.}
\end{table}
\item int32 \code{timestamps}: Array de longitud \code{timestamp\_count} con las etiquetas de tiempo guardadas al momento.
\item int32 \code{channel}: Array de longitud \code{timestamp\_count} con la máscara de bytes de los canales correspondientes a cada etiqueta de tiempo registrada.
\item int32 \code{coincWin}: Coincidence window. La ventana de tiempo para el conteo de coincidencias En múltiplos de \code{timebase}.
\item int32 \code{expTime}: Exposure time. El tiempo de integración para el cálculo de coincidencias. En milisegundos.
\item int32 \code{bincount}: Parámetro del histograma de diferencias temporales. Número de barras del histograma. En múltiplos de \code{timebase}, entre $1$ y $1,000,000$.
\item int32 \code{binsize}: Parámetro del histograma de diferencias temporales. Ancho de cada barra del histograma. En múltiplos de \code{timebase}, entre $2$ y $1,000,000$.
\end{itemize}

La lista completa de funciones implementadas para la paquetería de \code{hunahpy}, junto con su descripción, es:

\begin{enumerate}
\item error code \code{switch(byte)}: Función para debuggear el código. Esta fución se ejecuta después de casi todas las otras funciones y es la encargada de imprimir los códigos de error generados por cada función. En la Tabla \ref{switch} están todos los mensajes de error posibles y algunas notas sobre cómo corregirlos.

\begin{table}[] \label{switch}
\centering
\begin{tabular}{|l|l|l|}
\hline
\multicolumn{1}{|c|}{\textbf{Byte}} & \multicolumn{1}{c|}{\textbf{Mensaje}}                                                  & \multicolumn{1}{c|}{\textbf{Notas}}                                                          \\ \hline
-1                                 & Unspecified error.                                                                     &                                                                                              \\ \hline
0                                  & Success.                                                                               &                                                                                              \\ \hline
1                                  & Receive timed out.                                                                     & \begin{tabular}[c]{@{}l@{}}Se acabó la memoria para \\ cálculos.\end{tabular}                \\ \hline
2                                  & No connection was established.                                                         &                                                                                              \\ \hline
3                                  & Error accessing the USB driver.                                                        & \begin{tabular}[c]{@{}l@{}}Verificar que $\code{libusb0.dll}$\\  sea accesible.\end{tabular} \\ \hline
7                                  & \begin{tabular}[c]{@{}l@{}}Can't connect device because\\ already in use.\end{tabular} & Cerrar otras conexiones al TDC.                                                              \\ \hline
8                                  & Unknown error.                                                                         &                                                                                              \\ \hline
9                                  & \begin{tabular}[c]{@{}l@{}}Invalid device number used\\ in call.\end{tabular}          & No se estableció conexión al TDC.                                                            \\ \hline
10                                 & \begin{tabular}[c]{@{}l@{}}Parameter in func. call is out\\ of range.\end{tabular}     &                                                                                              \\ \hline
11                                 & Failed to open specified file.                                                         & \begin{tabular}[c]{@{}l@{}}Verificar que $\code{tdcbase.dll}$\\  sea accesible.\end{tabular} \\ \hline
\end{tabular}
\caption{Mensajes de error de un objeto TDC.}
\end{table}

\item error code \code{close()}: Cierra la conexión al módulo TDC. Sólo se permite una conexión a la vez - sólo un programa de adquisición abierto. Es importante cerrar la conexión antes de desconectarlo para evitar dañar el módulo.
\item error code \code{switchTermination(on)}: No confundirse con la función \code{switch()} que es la función de debugging. \code{switchTermination()} activa o desactiva el modo de impedancia de $50\Omega$ para los canales de entrada del TDC. Si está desactivado, la impedancia de los canales de entrada del TDC será $1000\Omega$. \code{on} es booleano: puede ser $1$ o $0$.
\item error code \code{configureSelfTest(channel,signal period,burst size,burst distance)}: El FPGA del TDC puede generar trenes de pulsos virtuales que serán procesados como señales añadidas a las entradas reales de cada canal.
\begin{itemize}
\item \code{channel}: Byte del canal a usarse (e.g. 5 corresponde a canales 1 y 3).
\item \code{signal period}: Período de todas las señales de un tren, en unidades de $20ns$. Entre $2$ Y $60$.
\item \code{burst size}: Número de períodos en el tren. Entre $1$ y $65,535$.
\item \code{burst distance}: Distancia entre trenes, en unidades de $80ns$. Entre $0$ y $10,000$.
\end{itemize}
\begin{figure}
\centering
\includegraphics[width=\linewidth]{imagenes/A1_selftest.png}
\caption{Señal de prueba del TDC.}
\label{fig:selftest}
\end{figure}
\item int8 \code{getChannel(byte)}: Regresa el número real de un canal basado en la máscara de bytes.
\item error code \code{getCoincCounters()}: Calcula el número de coincidencias para la última ventana temporal dado por \code{expTime}. Dos eventos son considerados coincidentes si suceden dentro de esa ventana. Este valor se guarda en la variable \code{coincidence\_array} en el orden: 1, 2, 3, 4, 5, 6, 7 ,8, 1/2, 1/3, 1/4, 2/3, 2/4, 3/4, 1/2/3, 1/2/4, 1/3/4, 2/3/4, 1/2/3/4. Las coincidencias no son acumulativas; sólo se guardan las coincidencias para la última ventana temporal de exposición.

\item error code \code{getDataLost()}: Verifica si hay pérdidas de datos causadas por una velocidad de detección demasiado alta para el cable USB. De haber datos perdidos, el valor de \code{data\_loss} queda como $1$ y se levanta un mensaje de aviso al usuario.
\item \code{getDeviceParams()}: Imprime los parámetros de operación del TDC.
\item error code \code{setHistogramParams(bincount, binsize)}: Define los parámetros del histograma de diferencias temporales \code{bincount} y \code{binsize}. Al llamarse esta función, se borran todos los histogramas anteriores acumulados hasta ese punto.
\item \code{getHistogramParams()}: Imprime los parámetros \code{bincount} y \code{binsize} del histograma de diferencias temporales.
\item \code{getHistogram(chanA,chanB,reset)}: Llama al histograma de diferencias temporales entre el canal \code{chanA} y el canal \code{chanB}.
\begin{itemize}
\item \code{chanA}: Primer canal del TDC para calcular el histograma. 0,...,7 para los canales 1,...,8. Si el valor está fuera de ese rango, calcula el histograma global.
\item \code{chanB}: Segundo canal del TDC para calcular el histograma. 0,...,7 para los canales 1,...,8. Si el valor está fuera de ese rango, calcula el histograma global.
\item \code{reset}: Si se deben borrar los histogramas anteriores acumulados hasta ese punto. Booleano: $1$ o $0$.
\end{itemize}
\item \code{getLastTimestamps(reset,output,*save)}: La función más importante de la paquetería. Guarda los valores de los últimos eventos procesados por el TDC en las variables \code{timestamps} y sus respectivos canales en \code{channels}. Si la cantidad de eventos procesados al momento de llamar la función es menor al tamaño total del búfer (el número total de eventos que el TDC puede almacenar en memoria en todo momento), la variable \code{valid} regresa el número total de etiquetas válidas en \code{timestamps} y \code{channels} (\textit{i.e.}, si el búfer tiene capacidad de $1000$ etiquetas pero sólo han sucedido $200$, \code{TDC.valid} tendrá un valor de $200$ y sólo las primeras $200$ entradas de \code{TDC.timestamps} serán no nulas).
\begin{itemize}
\item \code{reset}: Si se deben borrar los eventos anteriores acumulados hasta ese punto. Booleano: $1$ o $0$.
\item \code{output}: Si se deben guardar los eventos a archivo.  Booleano: $1$ o $0$. De ser positivo, llama a \code{saveTimestamps()}.
\item \code{*save}: Si \code{output = 1}, \code{*save} serán los argumentos que \code{saveTimestamps()} necesite: los nombres de archivo y extensión que serán creados.
\end{itemize}
\item \code{saveTimestamps(filename timestamps, filename channels,} \code{filename \\ extension)}: Guarda los valores de \code{timestamps} y \code{channels} en un archivo con nombre \code{filename timestamps} y \code{filename channels} respectivamente con extensión \code{filename extension}.
\end{enumerate}

La función \code{getLastTimestamps()} es la función central y más importante de la paquetería de \code{hunahpy} y permite guardar en la memoria de la computadora o en algún archivo de datos todos los eventos procesados por el TDC. Al crear un objeto \code{TDC}, se definen a partir del archivo \code{config.py} todos los parámetros necesarios por el TDC, como el tamaño del búfer A continuación, un ejemplo de lo que sería un programa sencillo para utilizar \code{hunahpy} desde un script de Python:

\begin{verbatim}
###################################
#from hunahpy import TDC          # Importa la librería de hunahpy
#                                 # para comunicarse con el TDC.
#                                 #
#tagger = TDC()                   # Esto inicia la conexión y deja 
#                                 # todo listo. En la consola, veremos 
#                                 # los mensajes de error para ver si 
#                                 # se conectó con éxito.
#...                              #
\end{verbatim}

En este punto, se puede realizar el experimento que vaya a generar los eventos para el TDC.

\begin{verbatim}
#...                              ###############
#tags_filename = "tags"                         #
#chan_filename = "chans"                        #
#file_extension = ".bin"                        #
#tagger.getLastTimestamps(reset=0,output=1,     #
#                         "tags","chans",".bin")#
#tagger.close()                                 #
#################################################
\end{verbatim}

Con esto, generamos dos archivos con las etiquetas temporales y sus respectivos canales en dos archivos \code{"tags.bin"} y \code{"chans.bin"}.
