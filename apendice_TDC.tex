\section{Paquetería de uso de \code{hunahpy}}\label{apendice_TDC}
La paquetería de \code{hunahpy} provee métodos útiles para el uso del módulo etiquetador de cuentas temporales \textit{id800}, fabricado por IDQuantique.

El nombre de \code{hunahpy} proviene de la cosmología maya narrada en el Popol Vuh: los dioses gemelos Hunahpú e Ixbalanqué, que fueron al inframundo por retar a los señores del Xibalbá y al emerger victoriosos se convirtieron en el Sol y la Luna.

La última finalidad de esta paquetería es ser utilizada para analizar correlaciones entre fotones gemelos, y el nombre \code{hunahpy} hace alusión a las semejanzas a la mitología de los dioses gemelos y su ascención a seres de luz. Eso y que la otra alternativa, \code{pyxbalanque}, sonaba muy fastidiosa.

\subsection{Introducción}
La biblioteca de uso del TDC consiste en varios métodos que aprovechan la funcionalidad del etiquetador para procesar señales de entrada. Hace uso de 3 bibliotecas compartidas:
\begin{itemize}
\item \code{libusb} y \code{nhconnect}, que proveen funcionalidad para reconocer y realizar conexiones a controladores USB.
\item \code{tdcbase}, librería compartida compilada en C por el fabricante del TDC que se comunica con el FPGA del etiquetador.
\end{itemize}

\subsection{Instalación}
Debido a que la biblioteca \code{tdcbase} está compilada en 32 bits, una distribución de 32 bits de Python es necesaria para poder correr el código.
Todos los archivos necesarios se encuentran en \textbf{github}. Una vez descargados, \textbf{FATLA SETUP PY}

\subsection{Uso}
Al importar \code{hunahpy}, se importa una sóla clase: \code{TDC}, que contiene todos los métodos necesarios, que se pueden llamar fácilmente.

\begin{verbatim}
from hunahpy import TDC
tagger = TDC()
...
tagger.getLastTimestamps()
tagger.close()
\end{verbatim}

